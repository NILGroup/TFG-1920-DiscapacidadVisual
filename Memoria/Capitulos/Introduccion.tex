\chapter{Introducción}
\label{cap:introduccion}

\chapterquote{Frase célebre dicha por alguien inteligente}{Autor}

Aquí podríamos incluir:

(donde digo ciegos quiero decir personas con discapacidad visual)
\begin{itemize}

	\item qué se considera por persona con discapacidad visual.
	\item datos sobre el número de personas ciegas que hay.
	\item datos sobre personas ciegas que usan un smartphone.
	\item problemas a los que se enfrentan los ciegos en el día a día (no solo referidos a navegación).
	
\end{itemize}

En la actualidad es prácticamente imposible imaginar  un \textit{smart phone} que no tenga instalado una aplicación de navegación.
Este tipo de aplicaciones se han convertido en herramientas esenciales para el día a día, puesto que no solo se limitan a dar una ruta entre dos puntos, sino a dar información sobre transporte público, horarios, opciones para el ocio...
Se estima que el $66,6\%$ de los usuarios utilizan aplicaciones de navegación mensualmente, con un crecimiento del $5,8\%$ desde 2017 hasta 2018. Y se espera que este crecimiento siga constante durante los próximos años \footnote{\url{https://www.emarketer.com/content/people-continue-to-rely-on-maps-and-navigational-apps-emarketer-forecasts-show}}. 


No cabe duda de lo útil que resulta poder consultar la ruta entre dos puntos. Pero, ¿estas aplicaciones resultan igual de apropiadas para todos? Alrededor del NOSECUANTOS PORCIENTO de los usuarios sufren algún tipo de discapacidad visual. Para ellas, estas aplicaciones son más importantes, si cabe. Puesto que enfrentarse a una ruta no conocida puede suponer todo un reto. 

Las personas con discapacidad visual se enfrentan cada día a numerosos desafíos, desde leer la etiqueta de un producto hasta saber si están en la parada de bus correcta. Bien es cierto que la tecnología ha avanzado mucho en este sector, en el \autoref{cap:estadoDeLaCuestion} veremos cómo algunas aplicaciones de navegación se han adaptado a estas necesidades. Sin embargo, aún vemos un claro hueco en cuanto a la navegación por interiores, este es un terreno menos explorado que la navegación por exteriores y, por ende, menos adaptado. Es en este sector en el que hemos centrado nuestro trabajo. 


El avance de las tecnologías en términos de adaptabilidad supone un incremento considerable en la calidad de vida de las personas que lo necesitan.  


\section{Motivación}
Ideas: Nos centramos un poco más en el problema de la navegación, podemos contar algunos aspectos de la entrevista que hicimos en la ONCE, sobre todo ese ``buuff'' que nos dieron por respuesta al preguntar por la situación de entrar en un edificio por primera vez.

Problemas más concretamente de navegación..
Que hay un vacío tecnológico en este tema, es un asunto que aún está por resolver.

IMPORTANTE: leerse las introducciones que hay en la carpeta de papers, porque hay cosas interesantes que se pueden meter aquí.
\section{Objetivos}
Descripción de los objetivos del trabajo.


\section{Plan de trabajo}
Aquí se describe el plan de trabajo a seguir para la consecución de los objetivos descritos en el apartado anterior.



\section{Explicaciones adicionales sobre el uso de esta plantilla}
Si quieres cambiar el \textbf{estilo del título} de los capítulos, edita \verb|TeXiS\TeXiS_pream.tex| y comenta la línea \verb|\usepackage[Lenny]{fncychap}| para dejar el estilo básico de \LaTeX.

Si no te gusta que no haya \textbf{espacios entre párrafos} y quieres dejar un pequeño espacio en blanco, no metas saltos de línea (\verb|\\|) al final de los párrafos. En su lugar, busca el comando  \verb|\setlength{\parskip}{0.2ex}| en \verb|TeXiS\TeXiS_pream.tex| y aumenta el valor de $0.2ex$ a, por ejemplo, $1ex$.

TFMTeXiS se ha elaborado a partir de la plantilla de TeXiS\footnote{\url{http://gaia.fdi.ucm.es/research/texis/}}, creada por Marco Antonio y Pedro Pablo Gómez Martín para escribir su tesis doctoral. Para explicaciones más extensas y detalladas sobre cómo usar esta plantilla, recomendamos la lectura del documento \texttt{TeXiS-Manual-1.0.pdf} que acompaña a esta plantilla.

