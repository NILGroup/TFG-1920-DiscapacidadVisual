\chapter*{Resumen}

%\section*{\tituloPortadaVal}
\section*{}

%Un resumen en castellano de media página, incluyendo el título en castellano. A continuación, se escribirá una lista de no más de 10 palabras clave.


Cada vez son más las aplicaciones de navegación disponibles capaces no solo de guiarnos entre dos puntos sino también de darnos información sobre el trayecto, como la congestión del tráfico, los horarios de un autobús, etc. Sin embargo, cuando nuestro destino está en el interior de un edificio las opciones de guía son escasas. El problema es aún más complejo cuando el edificio no es conocido, y se dispone de algún tipo de discapacidad visual que dificulta la lectura de carteles u otras señalizaciones.

Es por ello que en este trabajo se expone el desarrollo de una aplicación de guía en interiores adaptada a personas con discapacidad visual. Con ella se intenta paliar la sensación de desorientación que sufren estos usuarios cuando deben visitar por primera vez un edificio desconocido en su día a día. Se ha empleado, para ello, la Facultad de Informática de la UCM como edificio de estudio. Gracias al uso de balizas Bluetooth, esta aplicación es capaz de determinar nuestra posición origen dentro del edificio y guiarnos hasta el destino mediante instrucciones intuitivas y adaptadas por voz, que incluyen también información adicional sobre el recorrido (si hay puntos de interés cerca, como aseos, una cafetería, etc). Además, la aplicación está diseñada de tal manera que el manejo de la misma sea sencillo tanto para personas invidentes como videntes, abriendo a estos últimos la posibilidad de utilizarla.

Como se demuestra en la evaluación final de la aplicación, esta es completamente general e independiente del edificio en sí. Así, la aplicación puede desplegarse sobre cualquier edificio que se desee, independientemente del número de plantas o destinos que se quiera incluir.  De esta manera, todo lo que se requiere es una instalación de balizas Bluetooth y un dispositivo móvil con sistema operativo Android para continuar contribuyendo a la adaptación de nuevos espacios con la ayuda de este proyecto.

\section*{Palabras clave}
   
%\noindent Máximo 10 palabras clave separadas por comas
\noindent Navegación por interiores, baliza Bluetooth, beacon, posicionamiento, mapeo, discapacidad visual, ceguera, aplicación móvil
   


