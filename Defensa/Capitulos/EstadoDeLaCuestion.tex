\chapter{Estado de la Cuestión}
\label{cap:estadoDeLaCuestion}


En este capítulo revisaremos los aspectos necesarios para la comprensión de los modelos que se exponen en los capítulos que siguen. La Sección \ref{sec:cuestInmuno} brinda unas nociones básicas sobre inmunología, en las que se trata brevemente el estudio de los mecanismos y agentes del sistema inmune humano necesarios para la respuesta ante una infección, destacando el papel de las células T. Esta sección constituye una parte fundamental del trabajo, pues los modelos que se presentan a continuación deben ser entendidos a través del problema inmunológico que intentan explicar. Por su parte, la Sección \ref{sec:coop} habla sobre los modelos matemáticos en el campo de la biología y, más concretamente, sobre algunos de los que han abordado el problema de decisión entre división o apoptosis de las células T durante una infección aguda. 


\section{Cuestiones básicas de inmunología}
\label{sec:cuestInmuno}

Antes de comenzar es conveniente introducir una serie de definiciones y explicaciones básicas referentes al sistema inmune humano. De esta manera, los conceptos y modelos que se expondrán más adelante serán entendidos en su contexto y sin ningún impedimento terminológico.

En la Sección \ref{cap:introduccion} de Introducción ya decíamos que en el sistema inmune es posible observar comportamientos colectivos que son consecuencia de las decisiones individuales que toman sus células. Este sistema está compuesto por diversos agentes que trabajan de forma coordinada para dar una respuesta eficaz y proporcional al ataque recibido. Este último adjetivo es muy importante: necesitamos que la actuación de nuestro sistema inmune no sea insuficiente, lo que podría acarrear alguna inmunodeficiencia, ni tampoco excesiva, que es lo que ocurre, por ejemplo, con las alergias: el sistema inmune reacciona de manera exagerada a ciertos \textit{antígenos} que son, en la mayoría de casos, inofensivos. Otro de los requisitos que debe tener un buen sistema inmune es la capacidad para discriminar a quién hay que atacar y a quien no, evitando que las células del propio organismo sean blanco de su acción. Esto último es lo que sucede en el caso de las enfermedades autoinmunes, que pueden llegar a ser trastornos muy graves.

Describiremos brevemente a continuación los mecanismos de los que dispone el sistema inmune y cómo los utiliza. Haremos un recorrido desde lo más básico, comenzando por el \textit{sistema inmmune innato}, hasta conceptos más avanzados referentes al \textit{sistema inmune adaptativo}. Dedicaremos buena parte de esta sección a entender qué son las células T y cuál es su papel en el desarrollo de una respuesta ante una infección aguda. Como veremos, este tipo de células inmunes juega un papel primordial y, además, serán las grandes protagonistas de este Trabajo de Fin de Grado \citep{JTB}.  

\subsection{El sistema inmune innato}
\label{sub:sistInmInnato}

Comencemos por lo más simple: las barreras físicas. La piel y la mucosa de nuestro sistema respiratorio, digestivo y reproductivo intentan que virus, bacterias, hongos o parásitos no entren en nuestro organismo. Es la primera defensa que tenemos y es bastante efectiva en muchos casos, pero ¿qué pasa si estos agentes logran atravesar esta barrera?

Aquí entra en juego lo que se denomina \textit{sistema inmune innato} que, desde el punto de vista evolutivo, es el más antiguo de los sistemas inmunes de los seres vivos. De hecho, muchos mecanismos de este sistema inmune innato aparecieron hace más de $500$ millones de años \citep{theHowItWorks}. A pesar de que dispone de mecanismos mucho más sencillos que el \textit{adaptativo}, el papel que tiene es fundamental, pues permite dar una primera respuesta rápida ante una infección. 

Entre las armas de las que dispone encontramos proteínas, fagocitos y células NK (\textit{Natural Killer}), que son un tipo de linfocito producido en la médula ósea y que se distribuye por la piel, el intestino, el hígado, los pulmones y el útero, entre otros tejidos \citep{celulasNK}. Pero nos centraremos en uno de sus componentes más relevantes: los \textit{macrófagos}. Su nombre compuesto por dos palabras griegas: \textit{macro}, que significa grande y \textit{fago}, que significa comer, lo dice todo. En efecto, los \textit{macrófagos} son células que se comen invasores mediante un proceso llamado \textit{fagocitosis}, que ilustra la Figura \ref{fig:macrofago}. Durante la batalla con las bacterias, los \textit{macrófagos} producen y secretan unas proteínas llamadas citoquinas, que facilitan la comunicación entre células del sistema inmune y que cobrarán un papel muy relevante en los capítulos que siguen. Podríamos decir que los \textit{macrófagos} además de atacar a los invasores, pueden jugar el papel de centinelas, que cuando ven al enemigo mandan señales (citoquinas) para reclutar a más defensores. A continuación, veremos otros tipos de células, en este caso referentes al \textit{sistema inmune adaptativo}.

\begin{figure}[t]
	\centering
	\includegraphics[width=0.5\textwidth]{1_macrofago}
	\caption{Fagocitosis.}
	\label{fig:macrofago}
\end{figure}




\subsection{El sistema inmune adaptativo}
\label{sub:sistInmAdap}

Cuando el sistema innato no suficiente para detener el ataque de un patógeno, lo que ocurre por ejemplo si un virus logra evadir  a los macrófagos de guardia o  penetrar en células humanas para reproducirse en ellas, entra en acción el llamado sistema adaptativo, que está presente en humanos y vertebrados, pero no en la totalidad de los seres vivos.
 

Para explicar su funcionamiento necesitamos hacer uso de los conceptos de \textit{antígeno} y \textit{anticuerpo}. Los \textit{anticuerpos} son proteínas específicas que el cuerpo humano es capaz de producir y que pueden adherirse a otras sustancias, externas o internas, llamadas \textit{antígenos}. La misión principal de los \textit{anticuerpos} es identificar a los \textit{antígenos} generados por un agente patógeno, marcándolos así para su eliminación. Las células encargadas de la producción de \textit{anticuerpos} son las células B. Estas son un tipo de linfocito producido en la médula que, gracias a sus receptores de membrana, son capaces de identificar determinados complejos anticuerpo/antígeno para poder eliminar así a estos últimos. Cuando las células B nacen no están especializadas en la fabricación de un \textit{anticuerpo} concreto. Una vez que maduran, su ADN se recombina especializando así a la célula. Una vez que la célula B se encuentra con su \textit{antígeno} desencadenante, ésta produce muchas células grandes conocidas como \textit{células plasmáticas}. Cada \textit{célula plasmática} es esencialmente una fábrica para producir \textit{anticuerpos}. 

Es decir, gracias a la presencia de \textit{anticuerpos}, otras células, como los ya conocidos \textit{macrófagos}, son capaces de identificar a los elementos que hay que destruir cuando aún se encuentran en el medio extracelular como muestra la Figura \ref{fig:macrofago_anticuerpo}. Pero... ¿qué ocurre cuando un virus ya ha entrado en una célula de nuestro cuerpo? Los \textit{anticuerpos} no pueden alcanzarlo y el virus puede dedicarse a replicarse cuanto quiera. En este momento llega el turno de las protagonistas de este trabajo, las células T. 



\begin{figure}[t]
	\centering
	\includegraphics[width=0.6\textwidth]{2_macrofago_anticuerpo}
	\caption{Macrófago reconociendo una bacteria gracias a la acción \textit{anticuerpo-antígeno}.}
	\label{fig:macrofago_anticuerpo}
\end{figure}


\subsubsection{Las células T}
\label{Tcell}

Al igual que las células B, las células T se producen en la médula y ambas son muy similares en cuanto a su apariencia, de hecho, con un microscopio ordinario, un inmunólogo no sería capaz de diferenciarlas \citep{theHowItWorks}. La superficie de las células T también consta de unas moléculas que permiten la interacción con los \textit{antígenos} llamados receptores (TCR, \textit{T Cell Receptors}). Estos receptores permiten a estas células obtener información de su entorno y tomar decisiones en base a esa información. Por ejemplo, cuando los receptores de una célula T enlazan con un \textit{antígeno} compatible, las células proliferan para dar lugar a otras con la misma especificidad, es decir, que enlacen con el mismo \textit{antígeno}. Esta decisión de reproducción, que discutiremos con más detalle en los capítulos que siguen, es específica y lenta, tarda alrededor de una semana en completarse \citep{theHowItWorks}, lo que contrasta con la respuesta rápida que ofrece el \textit{sistema inmune innato}.

Hemos visto algunas de las similitudes que tienen las células B y T. Veamos algunas de sus diferencias: las células T maduran en el timo, de ahí la T de su nombre, mientras que las B maduran en la médula ósea. Además, las células B producen \textit{anticuerpos} que pueden reconocer cualquier molécula orgánica. Las células T, por su parte, están especializadas en el reconocimiento de un \textit{antígeno} específico y sus receptores permanecen siempre adheridos a la membrana celular y no pueden ser expulsados en forma de \textit{anticuerpo} como en el caso de las células B. Pero, quizá, la diferencia más importante sea que las células T no pueden reconocer al \textit{antígeno} ``por sí mismas'', necesitan que otra célula se lo presente \citep{theHowItWorks}. Las células que se encargan de ello se conocen como \textit{células presentadoras de antígeno}\footnote{Son macrófagos, células dendríticas, células B, entre otras.}. Las proteínas del microorganismo causante de la infección, una vez fagocitadas, son fragmentadas (formando los conocidos \textit{antígenos}) y transportadas hasta la superficie celular, donde quedan unidas a una estructura llamada \textit{complejo mayor de histocompatibilidad} (MHC) que se encuentra en la membrana de las \textit{células presentadoras de antígeno}. Gracias a su TCR las células T pueden reconocer aquellas células que han sido infectadas, puesto que el TCR y el MHC-péptido\footnote{Estructura formada por el MHC y el \textit{antígeno}.} encajan, la Figura \ref{fig:antigen_presentation} ilustra este proceso. Esta unión, si es perfecta, dura varias horas y se conoce como \textit{sinapsis inmunológica} \citep{fernandez2012mecanica}.



Hay distintos tipos de células T atendiendo al papel que desempeñan, los tres más importantes son: 
\begin{itemize}
	\item \textit{Killer o Cytotoxic T-Cells}: su misión es la de reconocer las células que han sido infectadas y, tras este proceso de reconocimiento, inducirlas al suicidio. De esta manera muere el virus, pero también la célula que había sido infectada por él. Constituyen una de las armas más potentes del sistema inmune.
	
	\item \textit{Helper T-Cells}: se encargan de regular la respuesta inmune. Una de sus tareas principales es secretar citoquinas para controlar que la respuesta inmune sea proporcional y las células T no reaccionen de manera descontrolada.
	
	\item \textit{Regulatory T-Cells}: estas mantienen la tolerancia a \textit{antígenos} propios, previniendo la aparición de enfermedades autoinmunes.
\end{itemize}



\begin{figure}[t]
	\centering
	\includegraphics[width=0.6\textwidth]{Imagenes/EstadoDeLaCuestion/Antigen_presentation}
	\caption{Proceso de activación de una célula T.}
	\label{fig:antigen_presentation}
\end{figure}


Cuando las células T salen del timo se encuentran desactivadas, en un estado \textit{naïve}, y se dedican a circular por los órganos linfoides secundarios, cuyos máximos representantes son los nodos linfáticos. Allí pueden encontrarse con \textit{células presentadoras de antígeno} provenientes del foco de una infección. Si las células T reconocen al \textit{antígeno} como extraño por medio de la \textit{sinapsis inmune}, se activan, convirtiéndose así en células efectoras, capaces de secretar citoquinas o de ir a la zona afectada a combatir la infección activamente. Una vez que las células han sido activadas, estas comienzan a proliferar masivamente, incrementando la población de células T activadas hasta en un factor de $10^6$ veces. En pocos días, las células pueden pasar por unos 15-20 ciclos de reproducción \citep{JTB}. Este proceso se conoce como \textit{expansión clonal}. Una vez que las células \textit{helper} han sido activadas pueden quedarse en los gánglios linfáticos, activando a otras células inmunitarias, o migrar al tejido infectado para secretar citoquinas y propiciar un ambiente adecuado para controlar la infección. Por su parte, las células \textit{killer} abandonan los gánglios linfáticos para identificar aquellas células infectadas en el organismo. Cuando el patógeno ha sido vencido, la mayoría de células T mueren, restaurando así los niveles de población iniciales (en caso contrario se acumularían millones de células que no son necesarias para el organismo) \citep{fernandez2012mecanica}. Este proceso se conoce como \textit{contracción clonal}. Sin embargo, es de gran utilidad conservar alguna de estas células experimentadas para poder reaccionar con rapidez en caso de que el mismo invasor vuelva a aparecer. Lo que hace nuestro sistema inmune es mantener un pequeño porcentaje de la población  ($5-10\%$) como células de memoria \citep{JTB}. Se llaman así porque guardan información del \textit{antígeno} contra el que combatieron. En caso de reaparición del patógeno, estas células se activan más rápidamente y nuestro cuerpo puede así generar antes una respuesta inmune.

A lo largo de este trabajo nos centraremos en el proceso de decisión entre división o suicidio celular de una célula T durante la respuesta inmune. En la sección y los capítulos que siguen veremos cómo se ha abordado este problema desde el punto de vista matemático y las conclusiones que su estudio ha permitido obtener. 


\section{Cooperación entre dos ciencias: matemáticas y biología}
\label{sec:coop}

En esta sección trataremos brevemente la interacción entre dos ciencias muy distintas: las matemáticas y la biología, y daremos algunos ejemplos de colaboraciones y modelos matemáticos creados para reproducir e investigar distintos procesos biológicos. Nos centraremos en aquellos referidos a las células T, sobre todo al caso que nos ocupa: la dinámica de población de las mismas durante la respuesta inmune.

Después de haber seguido un desarrollo independiente durante siglos, las matemáticas y la biología han comenzado a interaccionar activamente durante los últimos años. De hecho, los modelos matemáticos pueden llegar a ser una potente herramienta en el área de la biología. Como se puede leer en \cite{Gunawardena2014}, un modelo matemático es una máquina lógica que convierte hipótesis en conclusiones. Si el modelo es correcto y las hipótesis son ciertas entonces debemos, por lógica, creer sus conclusiones. Esta garantía lógica permite al matemático que desarrolla el modelo navegar con confianza lejos de las hipótesis y, probablemente, más lejos del lugar al que la mera intuición permite llegar.

Así pues, los modelos matemáticos son herramientas en las que un biólogo se puede apoyar, pero estos modelos deben tener ciertas características para poder considerarse de utilidad en Biología. A continuación, se presentan las guías que sugiere \cite{Gunawardena2014} para elaborar un buen modelo matemático:

\begin{enumerate}
	\item \textit{Formula una pregunta}. En ocasiones los modelos matemáticos no son diseñados para el avance del conocimiento de la biología, solo responden a investigaciones matemáticas que se basan, aparentemente, en problemas biológicos. Como ya se ha comentado en alguna ocasión, los modelos deben centrarse en aportar información que el biólogo desconocía. Intentar responder con un modelo a una pregunta puede ser clave a la hora de desarrollarlo con criterio, para que pueda ser juzgado por profesionales fuera del ámbito matemático. 
	
	\item \textit{Hazlo simple}. Incluir todos los procesos bioquímicos puede tranquilizar a los biólogos, pero no hará que el modelo sea mejor. De hecho, se convertirá en un modelo repleto de parámetros, poco flexible, difícil de estudiar y simular. Es mejor tener hipótesis simples y claras, intentando buscar una abstracción adecuada del problema.
	
	\item \textit{No es suficiente con que el modelo reproduzca hechos observados}. Si el modelo no puede ser comprobado ni refutado, entonces no está diciendo nada interesante. Ajustar los parámetros del modelo para que reproduzcan una observación no garantiza nada: un mismo fenómeno se puede representar mediante modelos distintos con parámetros diferentes

\end{enumerate}

Podemos distinguir dos tipos de estrategia en cuanto a los modelos se refiere: Modelado hacia adelante (\textit{forward modeling}) o inverso (\textit{reverse modeling}). El modelado inverso empieza con los datos experimentales, construye correlaciones entre ellos y les da estructura con un modelo matemático. Por su parte, el modelado hacia adelante empieza desde lo conocido, o sospechado, expresado en la forma de un modelo, a partir del cual se hacen predicciones. 

El modelado inverso se ha utilizado con el fin de analizar grandes volúmenes de datos genómicos y postgenómicos y, a veces, se equipara erróneamente con la biología de sistemas. Ocasionalmente ha sugerido nuevas ideas conceptuales, pero se ha utilizado con mayor frecuencia para sugerir nuevos componentes o interacciones moleculares, que luego han sido confirmados por enfoques biológicos convencionales. Los modelos en sí mismos han tenido menos importancia para comprender el comportamiento del sistema que como contexto matemático en el que la inferencia estadística se vuelve factible. En contraste, las mayores aportaciones a nuestra comprensión del comportamiento de problemas biológicos, como la homeostasis o la retroalimentación, han surgido del modelado hacia adelante. Puesto que los modelos actuales (cimentados en ecuaciones diferenciales o teoría de procesos estocásticos, por ejemplo) derivan, normalmente, de fenómenos y conocimiento conocidos. El primer beneficio que se obtiene de esto es que fuerzan al modelo a establecer unas hipótesis claras \citep{mathsModInmu}. Esto no implica que el modelado inverso no sea interesante. Hay muchas situaciones, especialmente cuando se tratan datos clínicos, donde la estructura de los datos se desconoce o es muy compleja, y las estrategias del modelado inverso cobran sentido \citep{Gunawardena2014}. 


A continuación, y teniendo presentes las observaciones anteriores, describiremos el punto de vista seguido en los artículos en los que se basa este trabajo.


\subsection{Dinámica de las células T. Decisión entre división o apoptosis}
\label{cuestionAmodelizar}

Recordemos brevemente el marco conceptual en el que nos movemos. En \ref{Tcell} decíamos que cuando las células T se activan en presencia de un \textit{antígeno} estas comienzan a reproducirse rápidamente para combatir la infección y, una vez superada, muchas de ellas se suicidan restaurando los valores de población iniciales. Es lo que denominábamos respectivamente como \textit{expansión clonal} y \textit{contracción clonal}. Más aún, los experimentos realizados ponen de manifiesto que la presencia del \textit{antígeno} no es suficiente para desencadenar la decisión de división o apoptosis, ya que las células T activadas continúan reproduciéndose incluso cuando el estímulo (\textit{antígeno}) está ausente y algunas se suicidan aun cuando la infección persiste \citep{JTB}. Estos son hechos observados; lo que se desconoce es el mecanismo de decisión por el cual una célula decide dividirse o morir. Varios modelos matemáticos, desarrollados bajo diferentes hipótesis, han sido propuestos para abordar este problema. Por una parte, se ha sugerido que el proceso de activación de las células T en estado \textit{naïve} desencadena un programa que solo depende de la estimulación por \textit{antígeno} inicial. Así las cosas, una célula T efectora (y por tanto, ya activada) comienza una serie de divisiones, desde un mínimo entre 7 y 10 y un máximo variable (relacionado con la estimulación por \textit{antígeno} que recibió cada célula de manera individual durante su activación) \citep{Hawkins5032}. Después de estas divisiones, la célula se suicida. Bajo esta suposición, la cantidad de \textit{antígeno} que percibe una célula T en estado \textit{naïve} durante su activación determina las divisiones de todas sus células hijas. Para precisar más este modelo, se propuso que este programa pudiera estar regulado también mediante citoquinas y no solo por la presencia de \textit{antígeno}, aunque los detalles concretos de esta regulación no son conocidos \citep{JTB}. Por otro lado, se han propuesto alternativas a este modelo basadas en procesos estocásticos. En este caso la decisión entre división o apoptosis de una célula T vendría determinada por la competición de dos relojes estocásticos \citep{DUFFY2012457}. Como ocurría en el caso anterior, los procesos celulares y moleculares específicos para dilucidar este algoritmo de decisión aún están en el aire. 

En esta memoria, presentamos otro modelo, expuesto en \cite{JTB}, cuyas hipótesis biológicas, ecuaciones y simulaciones se desarrollan durante los dos capítulos siguientes. Es un modelo que basa la decisión de cada célula T en la concentración de \textit{antígeno} y de dos proteínas inhibidoras, Retinoblastoma (Rb) y linfoma de célula B-2 (Bcl-2), que la célula encuentra en el medio extracelular que la rodea. Este algoritmo determinista distingue este modelo de los mencionados anteriormente y permite que cada célula decida, en función de la información que obtiene de su alrededor, la duración de su vida, si debe dividirse o no y el momento en el que debe hacerlo. Como veremos, esta anarquía a nivel individual se traduce en una propiedad emergente a nivel colectivo que concuerda con los hechos biológicos observados. 


