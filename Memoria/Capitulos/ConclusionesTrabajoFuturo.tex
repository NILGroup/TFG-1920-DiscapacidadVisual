\chapter{Conclusiones y Trabajo Futuro}
\label{cap:conclusiones}

En este capítulo se recogen las conclusiones obtenidas tras la realización y el análisis del proyecto. En la Sección \ref{sec:descResult} se detallan los resultados de los objetivos del proyecto (ver Sección \ref{sec:objetivosProyecto}). En la Sección \ref{sec:concluFinales} pueden verse las conclusiones finales del proyecto y, por último, en la Sección \ref{sec:trabajoFuturo} se abre el proyecto a nuevas colaboraciones con diferentes propuestas para trabajo futuro.

\section{Conclusiones}
\label{sec:concluFinales}

La idea de este proyecto surge de la necesidad de empoderar a las personas con discapacidad visual en el campo de las tecnologías. Pese a que hoy en día son cada vez más los que optan por desarrollar aplicaciones accesibles e inclusivas para este colectivo, aún es considerable la brecha existente en cuanto al mercado de aplicaciones que se encuentran adaptadas para su uso. Por ello, hemos decidido estudiar e indagar en el campo de las tecnologías accesibles y aportar nuestro granito de arena para ayudar a reducir las distancias a las que se enfrentan en el día a día. Con este fin, nuestra aplicación constituye una solución satisfactoria al problema de la navegación por interiores de manera general, y con una implementación específica para la Facultad de Informática, perfectamente adaptada e inclusiva.

El balance general de la aplicación es muy positivo ya que cumple los principales objetivos buscados. Entre sus puntos fuertes encontramos:
\begin{itemize}
	\item Da información sobre los metros que quedan hasta llegar a un punto determinado y a medida que se avanza en la dirección correcta se van actualizando disminuyendo.
	\item Utiliza indicaciones sencillas: ``continua recto'', ``gira a la derecha (o a la izquierda)'', etc.
	\item Avisa de las intersecciones mediante vibraciones.
	\item Avisa de la consecución del destino final mediante vibraciones.
	\item Cuando se supera una nueva indicación, la aplicación emite el sonido asociado a correcto o \textit{check}. En el caso de que la indicación no sea completada correctamente, la app informa y recalcula la ruta.
	\item Tiene una funcionalidad de \textit{instrucciones detalladas} que describe el entorno del usuario permitiéndole familiarizarse con él más rápidamente.
\end{itemize}

Por otro lado, el gran potencial de esta aplicación reside en que la hemos desarrollado de manera independiente al edificio original de modo que, si se siguen unas ciertas pautas de estructuración de la información, sea posible aplicar la guía a cualquier edificio. Para ello, es necesario contar con \textit{beacons} y distribuirlos por el interior del edificio, pues son los que determinan la posición del usuario. Sin embargo, gracias al sistema de posicionamiento empleado basado en puntos de decisión, esta instalación es menos rígida y por tanto facilita su integración en otros edificios. 

\section{Descripción de los resultados obtenidos}
\label{sec:descResult}
%USUARIO PERDIDO
El primer objetivo de este trabajo era conseguir generar una ruta completa de origen a destino. Para ello, teníamos que localizar a un individuo dentro del edificio de la Facultad de Informática empleando \textit{beacons}, proporcionar las instrucciones necesarias para la guía, detectar si un usuario va en la dirección adecuada o bien se ha perdido y en ese caso recalcular la ruta, etc. Este objetivo suponía un gran reto ya que añadía muchas novedades con respecto a los trabajos predecesores y atañía mucha complejidad. Además, el hecho de emplear por primera vez la tecnología \textit{beacon} en lugar de la señal Wi-Fi implicaba estudiar de cero su funcionamiento sin poder apoyarnos en investigaciones previas. Finalmente este objetivo ha sido completado satisfactoriamente y se ha desarrollado una guía completa por el interior de la Facultad de Informática.

De la mano de este objetivo surgió la necesidad de adaptar el mapeo de la facultad propuesto en trabajos anteriores al nuevo sistema de posicionamiento, precisamente porque ya no usaba Wi-Fi sino balizas Bluetooth. Esto ha supuesto una reestructuración completa de los cuadrantes implicando dibujar unos nuevos y modificar la información de los campos incluidos en los archivos XML. Esta adaptación se ha realizado con éxito, ya que no solo se ha ajustado perfectamente a la tecnología empleada sino que también se ha mejorado la información aportada y añadido nueva que nos ha permitido proporcionar instrucciones más detalladas y precisas que han supuesto una gran mejoría en la aplicación. Por otro lado, se ha llevado a cabo el mapeo de la planta baja, planta que difiere en estructura al resto de las del edificio, proponiendo de esta manera un mapeo más extenso de la facultad, que ha permitido añadir un mayor número de destinos y por ende, de rutas (incluyendo aquellas que van de una planta a otra).

Otro de los objetivos primarios fue el de conseguir adaptar la aplicación de manera que sea inclusiva y permita su uso a personas con discapacidad visual. Para ello se han propuesto diseños en consecuencia tanto en el cliente (en la interfaz, modo de uso y nuevas funcionalidades) como en el servidor (a la hora de generar una ruta más adecuada) y el mapeo (información relevante sobre el cuadrante y la conexión entre cuadrantes). Todo esto ha sido añadido al proyecto satisfactoriamente y el resultado ha sido una aplicación completamente inclusiva y accesible a personas con discapacidad visual independientemente del grado de dicha discapacidad. Esta aplicación además de aportar una guía de punto a punto incluye si se quisiera una descripción del interior del edificio a medida que se avanza.

Por último, se propuso un objetivo extra como consecuencia de la crisis provocada por la COVID-19, y que consistía en lograr que la aplicación fuese completamente independiente de la Facultad de Informática. Este objetivo ha supuesto adaptar el código, tanto del cliente como del servidor, para que la aplicación sea independiente del edificio y almacene todos los datos necesarios de archivos externos (estructurados de una manera concreta). Se ha conseguido así un código genérico que pueda ser reutilizado sobre cualquier edificio que siga las pautas indicadas: instalación de los \textit{beacons} en el edificio en cuestión y recopilación de la información necesaria relativa al mismo en archivos que sigan el modelo propuesto.

\section{Trabajo Futuro}
\label{sec:trabajoFuturo}

Tal y como está diseñada, la aplicación Blind Bit es fácilmente reutilizable en cualquier edificio público. Por lo tanto, animamos a proyectos futuros a que utilicen y extiendan la aplicación en espacios más ambiciosos como museos, aeropuertos o estaciones en los que todo tipo de personas pueden encontrar muy útil esta ayuda. Además, proponemos una serie de pequeñas mejoras que se pueden para mejorar la aplicación:
\begin{itemize}
	\item Especificación de los metros concretos de cada cuadrante en el XML (actualmente 5m por defecto debido a la imposibilidad de llevar a cabo mediciones reales en la Facultad por la crisis de la COVID-19). Estudio de la posibilidad de añadir un nuevo campo en el XML para indicar los metros tanto del ancho del cuadrante como del largo.
	\item Evaluación con usuarios finales invidentes para encontrar puntos mejorables prestando especial interés en los aspectos relativos a la interfaz, a las instrucciones proporcionadas, en cómo de intuitiva resulta, en funcionalidades que puedan echar de menos, etc.
	\item Instalación de los \textit{beacons} en los puntos señalados de la Facultad de Informática de la UCM, para convertirlo así en un edificio accesible para personas con discapacidad visual.
	\item Implementación de ajustes de configuración, entre los que proponemos la posibilidad de cambiar el idioma, el tipo de voz que lee las instrucciones (mujer por defecto), el volumen, la activación o desactivación del modo\textit{ Instrucciones detalladas}, etc.
	
	\item La inclusión de una base de datos que permita registrar y reconocer a distintos usuarios de manera que hagan \textit{Log In} y dentro de su sesión tengan una lista de destinos favoritos, sitios recientes, tengan guardada su configuración favorita en lugar de la por defecto, etc.
	
	\item Ampliación a una apliación más inclusiva de manera que no esté exclusivamente centrada en usuarios con discapacidad visual sino que este preparada y sea accesible para personas con otras discapacidades como por ejemplo la auditiva.
	
\end{itemize}

A pesar de que el código está pensado e implementado para dar servicio a un edificio de características similares a la Facultad de Informática de la UCM, también se ha pensado en posibles variantes que podrían surgir a la hora de adaptar el código a otro espacio. A continuación se plantean algunos casos y la solución propuesta: 

\begin{itemize}

	\item En el caso de la Facultad de Informática hemos establecido que cada cuadrante tiene un único punto de interés. Por ejemplo, los cuadrantes de los pasillos tienen un aula como punto de interés, pero podría ocurrir que en otra facultad hubiera aulas a ambos lados del pasillo y fuera, por tanto, necesario que un cuadrante tuviera más de un punto de interés. Esto se resolvería de manera sencilla adaptando la estructura del cuadrante en los archivos XML: incluyendo una etiqueta que estableciera la ubicación de cada punto de interés y adecuando en el código la lectura del mismo. En el archivo \textit{destinos.json} se incluiría una nueva entrada para el destino, de tal manera que tendríamos dos entradas con el mismo cuadrante (esto no supone un problema pues la clave de la tabla hash donde se almacena la información es el nombre del destino y no el cuadrante). Una vez modificados estos archivos bastaría con indicar a la función \textit{generar()} cuál es la ubicación del destino dentro de su cuadrante para que indique al usuario la posición del destino correctamente al finalizar la ruta.
	
	\item Otra situación que nos podemos encontrar es aquella en la que la estructura del edificio y los puntos que se quieran mapear se encuentren a una distancia demasiado próxima, obligando a que el tamaño de los cuadrantes se vea reducido. En este caso, dar instrucciones al usuario cada cuadrante puede resultar molesto, puesto que se darían instrucciones con demasiada frecuencia. Es por ello que se incluye una variable contador en la función \textit{generar()} que permite establecer el número de cuadrantes que queremos ``saltar'' antes de dar una nueva instrucción\footnote{En el propio código se han incluido en comentarios los cambios necesarios para la implementación de esta funcionaliadad.}, siempre que la dirección del usuario se mantenga estable. Es decir, no haya que hacer un giro, pues en ese caso el usuario debe ser advertido. 

\end{itemize}

A fin de que este proyecto pueda ser reutilizado, el enlace al repositorio público donde se encuentra el código, los mapas, archivos XML y demás documentos asociados a este trabajo puede verse a continuación: \url{https://github.com/NILGroup/TFG-1920-DiscapacidadVisual}