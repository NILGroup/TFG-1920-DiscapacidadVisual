\documentclass{article}
\usepackage[spanish]{babel}
\usepackage[utf8]{inputenc}
\usepackage{mathtools}


\title{\Huge Entrevista en el CTI de la ONCE}
\date{11 de octubre 2019}

\begin{document}
	
	\begin{titlepage}
		\maketitle
		\thispagestyle{empty}
	\end{titlepage}
	
	%ABSTRACT
	\begin{abstract}
		La idea de este TFG surgió de la necesidad de resolver problemas reales para gente real, en este caso para personas con discapacidad visual.
		
		Como no podía ser de otra manera, el camino comienza por lo más importante: conocer las necesidades de los usuarios finales.
		
		Gracias a la oportunidad que nos  brindó la Universidad, con la profesora María Guijarro al frente, pudimos entrevistar a personas con discapacidad visual y especializadas en el campo de las tecnologías.
		
		En este documento recogemos las notas que tomamos el día de la entrevista en el CTI (Centro de Tiflotecnología e Innovación) de la ONCE.
		
	\end{abstract}
	
	%PRIMERA SECCIÓN
	\section{Introducción}
		La entrevista comenzó con una breve explicación, de la mano de José María Ortiz (director del Departamento de Consultoría e Innovación), sobre las principales tareas que se llevan a cabo en el centro, entre las cuales destacan:
	
	 \begin{itemize}
		\item Ayudar en la adaptación  de una persona con discapacidad visual a su trabajo y vida cotidiana mediante la provisión de material necesario (teclados, líneas de braille, bastones, etc.).
		\item Responder a consultas sobre el funcionamiento de dispositivos.
	\end{itemize}
	
	Dentro de este centro encontramos diversos departamentos:
	
	\begin{itemize}
		\item El departamento de Consultoría e Innovación: donde desarrollan el programa EDICO en colaboración con la UCM. Este tiene como objetivo hacer las matemáticas accesibles mediante un editor de texto. 
		De manera paralela se encargan del desarrollo de aplicaciones para la biblioteca de la ONCE, películas audio-descritas, etc.
		\item Departamento de Evaluaciones y Auditoría: donde se encargan de evaluar los productos que se van a sacar al mercado.
		\item Departamento de diseño y producción: donde se encargan de, tal y como indica su nombre, diseñar y producir elementos de adaptabilidad, como pueden ser unas plantillas con relieve de policarbonato para las vitrocerámicas. Recordemos que estas, aunque no presentan dificultad alguna para la mayoría de usuarios videntes, son tediosas para aquellos que cuentan con discapacidad visual ya que la pantalla táctil no tienen ningún tipo de relieve que pueda servirles como referencia y guiarles en su uso.
		\item Departamento de Asesoría en tecnología: especializado en tecnologías accesibles.
	\end{itemize}
	
	
	%SEGUNDA SECCIÓN
	\section{Preguntas}
	
	Comienza la ronda de preguntas. Por turnos, los distintos grupos comenzamos a hacer preguntas a Mónica ¿? y José Luis Llorente (ingenieros del CTI). Además, Mónica es invidente, por lo que nos dió su perspectiva también desde el punto de vista del usuario.
	
	\begin{itemize}
		\item  \textbf{¿Cómo utiliza una persona con discapacidad visual un dispositivo móvil?}
		
		Para responder a esta pregunta, Mónica nos hace una demostración en directo, para ello emplea un móvil Xiaomi con sistema operativo Android.
				
		Para la navegación por su dispositivo, Mónica nos cuenta que utiliza un lector de pantalla, es decir, un software que facilita el uso del sistema operativo. Éste sirve como guía para las personas que, como ella, tienen discapacidad visual, ya que ``lee y explica'' mediante voz lo que se visualiza en la pantalla. Los lectores de pantalla vienen siempre incluidos en el dispositivo y se pueden encontrar en la sección de Accesibilidad, en Ajustes. En el caso de Android, este software se llama \textit{Talkback} y es configurable. Por ejemplo, dice Mónica, se podría usar mediante la línea de braille en vez de la reproducción por voz.
		\\
		\\
		Luego vemos como se desplaza por las aplicaciones utilizando \textit{flicks}, movimientos secos en los que desliza el dedo hacia uno de los lados de la pantalla (izquierda o derecha, según interese). Del mismo modo, para la navegación por la web o dentro de alguna aplicación utiliza estos movimientos hacia arriba y hacia abajo. Por último, nos muestra cómo accede a un elemento mediante doble click. 
		\\
		También nos habla de la posibilidad de la navegación libre, eso sí, solo cuando ya te has familiarizado con el dispositivo lo suficiente como para saber dónde tienes determinadas aplicaciones. 
		\\
		\\
		Lo más cansado, según Mónica, es tener que hacer un barrido por toda la pantalla hasta encontrar lo que quieres, en vez de poder ir directamente. Para agilizar un poco este proceso, Mónica, por ejemplo, agrupa las aplicaciones por carpetas, de modo que el barrido es más sencillo que si la pantalla estuviese repleta.
		\\
		\\
		Para las personas con baja visión también existe la posibilidad de hacer más grandes los iconos y ajustar los colores.
		\\
		
		% Segunda pregunta:
		\item \textbf{Hemos leído que normalmente las aplicaciones se desarrollan para dispositivos iOS, ¿por qué es mejor?} 
		\\
		\textit{"Si que es cierto que solía ser así ya que iOS le llevaba la delantera a Android en cuanto a accesibilidad, pero cada vez se usa más Android pues las diferencias están completamente recortadas, están muy igualados y los precios son mucho más asequibles. Yo misma, antes tenía un iPhone y ahora me he pasado a Android y no hay nada que eche en falta."} Nos comentaba Mónica al respecto.
		
		% Tercera pregunta:
		\item \textbf{¿Cómo afronta una persona ciega su desplazamiento y orientación por interiores cuando pisa por primera vez dicho espacio u edificio?}
		\\
		Ante esta pregunta Mónica resopló y nos contestó con un \textit{Buufff..., ¿te vale?}
		\\
		Nos puso como ejemplo la llegada a un hospital: \textit{"cuando entras necesitas saber, al menos, dónde está la recepción para pedir ayuda pero los carteles informativos están fuera de mi alcance, entonces entro por la puerta y pienso ¿y ahora qué?. ¿Dónde está el mostrador de recepción? 
		\\
		No es tan fácil como echar un primer vistazo, necesitas ayuda mediante voz, algo que te describa el espacio y te vaya diciendo que hay a derecha e izquierda y a cuantos metros."}
		\\
		\\
		Nos comentó que en cuanto a la descripción/guía por espacios interiores ahora mismo no se dispone de ninguna aplicación. Por ello, una vez superada la primera barrera de ubicar y localizar un cierto destino, la única opción que les queda es la de memorizar el camino. Mónica decía sorprendida que era increíble la cantidad de rutas que tiene en la cabeza.
		\\
		\\
		Es por todo esto que una aplicación sería de gran ayuda para ellos, de manera que pudiesen tener una idea del edificio incluso antes de llegar a él, para moverse con más seguridad. Una app que no solo les guiase a un punto concreto, sino que además describiese el edificio, indicándoles que posibilidades les ofrece.
		\\
		\\
		También se hizo mención de otras propuestas e ideas como tener previamente el plano del edificio para poder ir moviéndote con el dedo sobre él y que te vaya indicando las distintas salas que ofrece; o la impresión de un mapa 3D que disponga de un código QR o algo similar (que fuese capturado mejor por Bluetooth que por foto) que tras leerlo cargue el plano del edificio y pueda proporcionar tanta información sobre el espacio en sí: número de plantas, qué hay en cada una...Como información más precisa como puede ser que uno de los baños esté averiado, etc.
		\\ 
		\\
		Obviamente de la mano de estas ideas surgían problemas y controversia: ¿Dónde estaría dicho mapa?, ¿Cómo encontrarlo?, ¿Todos los edificios facilitarían los planos o puede que por motivos de seguridad no sea una idea muy factible?, ¿sería posible llegar a un standard para que se pudiera usar el mismo sistema en cualquier edificio?
		\\
		 
		% Cuarta pregunta:
		\item \textbf{¿Hay algún tipo de señales que os sirvan como referencia a la hora de desplazaros por un edificio?}
		\\
		\textit{Hay señales de encaminamiento, que te indican dónde están las escaleras, ascensores, zonas de cruce, etc.}
		\\
		% Quinta pregunta:
		\item \textbf{¿Cuántos edificios cuentan con estas señales?}
		\\
		\textit{La verdad que cada vez son más frecuentes y hoy en día se encuentran en casi todos los edificios, especialmente en los nuevos.}
		\\
		% Sexta pregunta:
		\item \textbf{¿Cómo de factible es ir con el dispositivo móvil en la mano, para realizar una foto o cualquier cosa similar?}
		\\
		\textit{Puedo hacer una foto en un momento puntual, en eso no hay problema alguno pero no es cómodo ir con el móvil en la mano constantemente porque además de que puede ser aparatoso ya que ya llevo en la mano el bastón, perro guía, etc. No es práctico pues no sería la primera vez que roban un móvil a una persona invidente, es una realidad. 
		\\
		Particularmente, con respecto a la foto el problema principal sería saber a dónde enforcar.} 
		 
	\end{itemize}


	%TERCERA SECCIÓN
	\section{Conclusiones}
	Algunas conclusiones que sacamos de la visita al CTI:
	
\begin{itemize}
	\item La implementación de una aplicación como la nuestra es muy útil y necesaria.
	\item Tienen varias maneras de interactuar con el móvil: flicks, sacudiéndolo, por vibración, arrastrando o pulsando con un dedo, dos,...
	\item Ir barriendo el espacio con la cámara del móvil no les resulta cómodo.
	\item El uso de dispositivos adicionales como una micro cámara, en principio, no sería un problema, siempre y cuando no lo tengan que llevar siempre en la mano, pues esto les desencadenaría una serie de problemas.
	\item Si se utilizan cascos, estos deben dejar el canal auditivo libre, para que puedan oír otros estímulos (auriculares óseos).
	\item Se trata de que las aplicaciones (en general) sean lo más inclusivas posibles, es decir, que su uso sea apto para personas videntes y aquellas con discapacidad visual.
	\item El feedback que dé la aplicación no debe suponer un dolor de cabeza pero está bien que sea constante para que no piensen que la aplicación ha dejado de funcionar.
\end{itemize}
	
	
\end{document}