\chapter{Diseño e implementación}
\label{cap:diseñoeimplementación}

Hemos planteado la aplicación como un modelo cliente-servidor, en la que el cliente (dispositivo móvil) se encarga de la conexión con el usuario, es decir, se ocupa de solicitar los datos necesarios (origen, destino, posición actual, etc.) para enviárselos al servidor. Una vez que el servidor tiene esta información procede al cálculo de la ruta (obtención de la lista de cuadrantes y de instrucciones) para posteriormente proporcionársela al cliente y que sea este el encargado de comunicársela al usuario. De esta manera, nuestra aplicación está mejor organizada, es más eficiente y evita que los dispositivos móviles se queden sin batería rápidamente a causa de una pesada carga computacional. A continuación detallaremos el diseño y los detalles técnicos de la implementación tanto del cliente como del servidor.


\section{Servidor}
El servidor constituye una parte indispensable del proyecto ya que se encarga de realizar los cálculos más pesados para no sobrecargar al dispositivo móvil. Sus principales funcionalidades son:
\begin{itemize}
	\item Almacenar la estructura e información referente al edificio mapeado.
	\item Permanecer a la escucha de cualquier cliente que solicite conexión.
	\item Solventar el posicionamiento del cliente conectado.
	\item Generar la ruta óptima desde la posición actual hasta el destino indicado por el cliente.
\end{itemize} 

La aplicación servidor está diferenciada en dos partes: el código escrito en lenguaje Java y los archivos correspondientes al edificio que se mapea, en nuestro caso la Facultad de Informática de la UCM. Estos archivos son los xml mencionados en la Sección \ref{sec:mapeo} y un archivo json que contiene la información referente a los distintos cuadrantes asociados a los destinos considerados por nuestra aplicación (por ejemplo, secretaría, conserjería, cafetería, las aulas,  etc.). 
Buena parte del código que conforma el servidor ha sido reutilizado de trabajos anteriores, concretamente de los proyectos de TFG \textit{Generador interactivo de instrucciones de guía sobre plataformas móviles} \citep{TFGguia} y del proyecto \textit{Sistema de guía por voz en interiores} \citep{TFGMariana}. La estructura de nuestro servidor se corresponde con la de \cite{TFGguia}. Sin embargo, hemos introducido cambios notorios para el desarrollo de esta aplicación que comentaremos a continuación.

\subsection{Funcionamiento del servidor}

Para entender mejor la estructura del servidor, haremos un recorrido desde la carga de la información procedente de los archivos xml que representan el mapeo del edificio hasta la conexión con el cliente y el cálculo de la ruta: 

\begin{itemize}
	\item \textit{Arranque del servidor:} Cuando el servidor arranca es necesario que guarde la información relativa al edifico para que pueda almacenar su estructura y generar una ruta válida para proporcionársela al cliente que la solicita. Al igual que los xml, el código relacionado con la carga de estos archivos, que conforma las clases \textit{CargaXML} y \textit{Edificio}, también se apoya en el del proyecto de \citep{TFGguia} ya mencionado en otras ocasiones. Estas clases permiten cargar los archivos y almacenar la información referente a los cuadrantes estructurada en plantas. Para nuestro trabajo se han añadido los atributos de clases necesarios para guardar la información nueva incluida en los xml (como los \textit{beacons}, los metros que ocupa el cuadrante o la información relevante de este) y se han eliminado aquellas que ya no se utilizan (como las coordenadas sureste y noroeste que no están presentes en nuestro proyecto). Una vez que está cargada esta información y ya tenemos la lista de cuadrantes existentes, pasamos a generar la matriz de adyacencia de la clase \textit{ListaCuadrantes} con la que representamos el mapa en forma de grafo y establecemos las conexiones entre cuadrantes. Esta tarea es relativamente sencilla, pues los xml nos proporcionan la información sobre los cuadrantes colindantes. El cambio más notorio que se ha introducido en este punto es dotar a las conexiones de la matriz de adyacencia de determinados pesos según la adaptabilidad de la ruta a nuestros usuarios. Estos cambios aparecen explicados con detalle en la Sección \ref{sub:rutaOptima}. 
	
	Es en este momento inicial cuando el servidor también carga el archivo \textit{destinos.json} con la lista de destinos y los cuadrantes asociados correspondientes. Una de sus entradas es, por ejemplo: \{``lugar'': ``aula 9'', ``cuadrante'': ``4''\}. En la clase \textit{LectorDestino} (reutilizada del proyecto de TFG \citep{TFGguia}) se almacenan cada una de las entradas en una tabla hash que cuenta con un campo clave de tipo String y un campo valor de tipo Integer que relaciona el lugar con su cuadrante asociado.
	
	\item \textit{Conexión con el cliente:} Una vez que el servidor ha almacenado toda la información correspondiente al edificio está preparado para recibir peticiones de los clientes. El servidor queda entonces a la espera de los clientes en la clase \textit{MainClienteAndroid}, escuchando a través de un \textit{webSocket} en un puerto determinado. Esta conexión cliente-servidor constituye otro de los cambios principales realizados con respecto a los proyectos anteriores, pues se ha reestructurado por completo. En primer lugar, la conexión ya no se hace por medio de \textit{sockets} sino por medio de \textit{webSockets}, esto implica una mayor seguridad en el intercambio de mensajes (se emplea el protocolo http) y permite que el código del servidor pueda ser utilizado como servidor externo. En nuestro caso hemos montado el servidor sobre una máquina virtual con una IP pública de la Facultad de Informática utilizando TomCat\footnote{\url{http://tomcat.apache.org/}}. Esto permite que los clientes puedan acceder al servidor desde cualquier red, lo cual es primordial para poder establecer conexión con el servidor desde la propia Facultad en particular. En segundo lugar, el número de conexiones que tiene que establecer el cliente con el servidor para obtener la información necesaria de la ruta se ha optimizado al máximo. En un solo mensaje el servidor envía al cliente toda la información necesaria. En el proyecto que hemos tomado como modelo el cliente en cambio, debía solicitar al servidor una nueva instrucción cada vez que actualizaba su posición. Ahora el cliente hace una única petición indicando su \textit{beacon} más cercano (que se tomará como origen) y el destino al que quiere ir, en un mensaje del tipo \textit{IDdelBeacon$|$destino}, por ejemplo: \textit{CPne$|$aula 8}.	Cuando el servidor recibe este mensaje, se encarga de generar la ruta desde el origen hasta el destino y enviársela al cliente. La información que se envía está compuesta por la lista de beacons asociados a los cuadrantes que conforman la ruta desde el origen hasta el destino, las instrucciones necesarias, la lista de \textit{booleanos} que indican cuándo hay que hacer un giro en la ruta y la información adicional de los cuadrantes que conforman la ruta. Por ejemplo, para la petición del cliente \textit{beacon34$|$aula 3}, donde suponemos que \textit{beaconX} indica que es el \textit{beacon} del cuadrante $X$ y que el aula 3 es el cuadrante 24 (ver Figura \ref{fig:ruta_optima}, línea verde), obtendríamos el mensaje que vemos en la Figura \ref{fig:ejemplo_ruta}. Donde la lista de \textit{beacons} se representa en verde con un centinela FINAL que indica que en el \textit{beacon24} se termina la ruta, la lista de instrucciones a seguir en azul, separadas por el carácter @ atendiendo al cuadrante al que pertenecen. Debido a la imposibilidad de ir a la Facultad de Informática a causa del COVID-19 se ha supuesto que todos los cuadrantes miden cinco metros, excepto el $36$ que mide 10, esta información no se corresponde con la realidad. En morado se representa el cuadrante en el que hay que girar, como vemos el ``sí'' corresponde al cuadrante $20$, que es en el que hay que hacer el giro. Por último, en gris se presenta la información adicional de cada cuadrante, un ``no'' indica que no hay información asociada a dicho cuadrante. %HAY QUE PONER AQUÍ UNA RUTA IGUAL MÁS CHULA CON LA INFORMACIÓN BIEN PERO PARA QUE SE VEA LO QUE HACE DE MOMENTO.
	
	  
	\item \textit{Generación de la ruta:} Esta funcionalidad es la más importante del servidor. El objetivo es obtener toda la información referente a la ruta desde el origen al destino seleccionado (ver Figura \ref{fig:ejemplo_ruta}). Lo primero que hay que hacer es obtener los cuadrantes origen y destino, recordemos que lo que tenemos es el \textit{beacon} más cercano al cliente y el destino como cadena de caracteres (String), no como cuadrantes. Para ello, se busca el cuadrante asociado al \textit{beacon} y al destino recorriendo la lista de cuadrantes del edificio (\textit{aCuadrantes}) y solicitando el valor asociado al destino (clave) en la tabla hash (\textit{lectorDest}). Con esta asociación entre \textit{beacons}/destinos y cuadrantes se resuelve el problema del posicionamiento. 
	
	Una vez que tenemos los puntos que determinan la ruta se genera la lista de cuadrantes y de \textit{beacons} correspondientes mediante la función \textit{calculaRuta}. Esta está basada en el algoritmo de \textit{Dijkstra} (que reemplaza a la búsqueda en anchura utilizada en el proyecto de \citep{TFGguia}) que tiene como entrada la matriz de adyacencia de los cuadrantes. Una vez que tenemos la lista de cuadrantes por los que el usuario debe pasar hasta llegar al destino se generan las instrucciones necesarias para guiar al usuario. Este es el cometido principal de la función \textit{generar}. Esta función tiene como entradas el cuadrante actual (en el que se encuentra el usuario), que se va simulando mediante un bucle, y el destino, y en función de los cuadrantes de la ruta, más concretamente de los inmediatos al cuadrante actual, construye la siguiente instrucción (ver detalles en la Sección \ref{sub:genInstruc}). Una vez que se ha llamado a \textit{generar} con todos los cuadrantes que conforman la ruta, ya se tiene la lista de instrucciones completa y se contesta al cliente.
\end{itemize}


\begin{figure}[t]
	\centering
	\includegraphics[width=0.9\textwidth]{Imagenes/Capitulo4/ejemploRuta}
	\caption{Ejemplo de la información generada por el servidor para una ruta desde el cuadrante $34$ al $24$.}
	\label{fig:ejemplo_ruta}
\end{figure}


\subsection{Cálculo de la ruta óptima}
\label{sub:rutaOptima}

En esta sección veremos las modificaciones que se han hecho para lograr guiar al usuario por la ruta más conveniente. Como ya hemos mencionado anteriormente, el mapeo nos proporciona un grafo en el que los cuadrantes son los nodos y las conexiones entre ellos, las aristas. Para su representación hemos empleado una matriz de adyacencia y de esta manera, el cálculo de la ruta más corta entre dos cuadrantes se reduce al algoritmo de \textit{Dijkstra}.

Sin embargo, no debemos olvidar que nuestra aplicación tiene un usuario final muy concreto: personas con discapacidad visual. Es por ello que la ruta debe ser lo más sencilla posible, libre de obstáculos y otros elementos que puedan entorpecerlos, por lo que en ocasiones la ruta óptima no coincide con la más corta sino con la que esté más adaptada. Para representar esta adaptabilidad, a aquellas conexiones que presenten una mayor dificultad para las persona invidentes se les ha asignado un peso mayor, a fin de buscar este equilibrio entre la ruta más corta y la más adecuada. En la Figura \ref{fig:ruta_optima} aparece un ejemplo de esta situación: si quisiéramos ir desde el aula 3 (cuadrante $24$) a la puerta principal (cuadrante $34$), el camino más corto implicaría pasar por delante de los ascensores (ruta roja). Sin embargo, la conexión entre los cuadrantes $31$ y $32$, y $22$ y $31$ puede resultar tediosa para una persona invidente. Las razones que hemos considerado son: 
\begin{itemize}
	\item El tramo compuesto por los cuadrantes $31$, $32$ y $22$ es más estrecho que su camino paralelo por el \textit{hall}.
	\item Presenta más giros e intersecciones de caminos.
	\item Normalmente acumula más gente ya que es zona de paso para coger los ascensores o subir/bajar por las escaleras o entrar/salir de la cafetería.
	\item Obliga a pasar por dos pequeños peldaños o una rampa estrecha.
\end{itemize} 
 	Todo esto es potencialmente problemático en ausencia de la vista y considerando además que nuestros usuarios es probable que vayan acompañados de un perro guía o un bastón. Por ello, hemos concluido ajustar los pesos en la matriz de adyacencia y la ruta generada pasa a ser la señalada en verde. De esta manera limitamos el paso por el pasillo de los ascensores a aquellas rutas en las que es estrictamente necesario, es decir, cuando se requiere un cambio de planta.


\begin{figure}[t]
	\centering
	\includegraphics[width=0.8\textwidth]{Imagenes/Capitulo4/mapa_ruta_optima}
	\caption{Ejemplo de ruta óptima entre dos puntos.}
	\label{fig:ruta_optima}
\end{figure}

\subsection{Generación de instrucciones}
\label{sub:genInstruc}

La función que contiene toda la lógica relativa a la generación de las instrucciones es \textit{generar}. Esta función es una de las que más modificaciones ha sufrido con respecto a los trabajos predecesores pues no solo la hemos adaptado para personas con discapacidad visual sino que también hemos incluido mucha complejidad procedente de los cambios de planta y demás casuística que no estaba previamente incluida. Algunas de las modificaciones son: 

\begin{itemize}
	\item Se ha añadido más precisión e información a las instrucciones. De esta manera, cuando el usuario llega al destino correspondiente la instrucción especifica dónde se encuentra este, véase ``Su destino está a la derecha (o a la izquierda o delante...).'' dependiendo de por dónde haya llegado el usuario. Otro caso en el que se ha añadido mayor precisión es al puntualizar los metros que el usuario debe continuar en una dirección dada. En la Figura \ref{fig:ejemplo_ruta} podemos ver un ejemplo de ruta en el que se especifica el número de metros que el usuario debe seguir recto hasta ejecutar el siguiente giro o la siguiente acción, y como a medida que el usuario va avanzando la distancia va disminuyendo. Esto favorece que el usuario reconozca que va en la dirección correcta y pueda calcular mejor cuando ha de girar o ejecutar cualquier otra acción. A diferencia de los trabajos predecesores, nuestra app proporciona una instrucción nueva en cada cuadrante, es decir, no se salta ninguno como sucedía en el caso de las rectas en el proyecto precedente. Esto se ha hecho así ya que por un lado hemos considerado cuadrantes más grandes y por tanto la distancia recorrida de un cuadrante a otro es mayor, y porque creemos muy conveniente que el usuario reciba de manera continuada \textit{feedback} por parte de la aplicación para cerciorarse de que va por el camino correcto y que la aplicación no ha dejado de funcionar. Finalmente, también se ha ajustado el orden de las instrucciones para proporcionarlas de manera correcta. De modo que si la instrucción siguiente es un giro se invierte el orden para indicar primero el giro y después los metros que debe continuar en la nueva dirección.
	
	
	\item Se han incluido cambios de planta, esto supone una gran novedad respecto al proyecto citado \cite{TFGguia}. Ahora los cuadrantes de distintas plantas correspondientes a los ascensores están unidos\footnote{El cuadrante $31$ (correspondiente a los ascensores situados detrás de conserjería) está conectado con el $18$ (ascensor correspondiente al anterior en la planta 1) y el $29$ (ascensor más cercano a la puerta trasera de la cafetería) con el $10$ (ascensor correspondiente al anterior en la planta 1).} permitiendo así, que la lista de cuadrantes de la ruta esté formada por cuadrantes de distinta planta. La función \textit{generar} detecta cuando el siguiente cuadrante al que queremos ir no está en la misma planta que el de nuestra posición actual y genera la instrucción en consonancia ``Los ascensores están a tu izquierda. Sube a la primera planta.'' o ``El ascensor está delante. Sube a la primera planta.'', por ejemplo. También se tiene en cuenta en las instrucciones el caso particular de las zonas de ascensores, pues hay dos zonas por planta pero no son simétricas por lo que las instrucciones son distintas. Por ello, para salir de la zona de los ascensores correspondientes al cuadrante $31$ (los que se sitúan detrás de conserjería) debemos simplemente girar, mientras que para salir de los de la zona de la puerta trasera de la cafetería (cuadrante $29$) debemos caminar escasos metros hacia adelante y después girar. A pesar de que en la Facultad de Informática contamos con ascensores y escaleras, se ha supuesto que la ruta se seguirá por medio de los ascensores ya que de esta manera es más fácil el cambio de planta y, además, estos se encuentran adaptados con botones en \textit{braille}.
	
	\item Se ha añadido una funcionalidad que permite informar al usuario (si lo desea) sobre lo que se va encontrando a su paso por la ruta véase los baños, la cafetería, el despacho de Delegación de Alumnos, unos escalones, etc. Esto es fundamental pues además de proporcionar seguridad al usuario en su primera visita a un edificio, le da la opción de tener una mejor idea global del espacio en el que se encuentra. Para ello lo que se hace es incluir en la ruta la información contenida en el cuadrante siguiente al que se encuentra el usuario, de esta manera conseguimos adelantarnos y avisar al usuario con antelación. Así mismo, la función \textit{generar} enviará información sobre cuándo el usuario deberá hacer un giro para permitir avisar a este de manera especial (con una vibración) desde la aplicación del cliente.
	
\end{itemize}


%Comento la figura del diagrama de clases porque quiero ver cómo lo reorganizo
%\begin{figure}[t]
%	\centering
%	\includegraphics[width=1.1\textwidth]{Imagenes/Capitulo4/diagramServer}
%	\caption{Diagrama de las clases principales del servidor.}
%	\label{fig:diagServ}
%\end{figure}



\section{Cliente}

El cliente de nuestro proyecto está constituido por la aplicación Blind Bit desarrollada para Android. Con esta aplicación podemos conectarnos al servidor e iniciar la ruta al destino deseado desde el punto en el que nos encontremos (siempre que este sea el exterior de las aulas u otras estancias). La aplicación se encarga de darnos, en el momento adecuado, las instrucciones necesarias para llegar al destino y va avisándonos mediante sonidos o vibraciones de diferentes acciones como giros o aprobación de que seguimos en el camino correcto. A continuación presentamos tanto los detalles de diseño de la aplicación como los detalles técnicos de su implementación. 


\subsection{Diseño de la aplicación Blind Bit}

%Aquí hablar de la interfaz y justificar el por qué de las decisiones tomadas (el título de las pantallas, posición y forma de los botones, soniditos y vibraciones, etc) + fotos de la interfaz.

INTRODUCCIÓN (no estoy inspirada)

A continuación describimos las pantallas de nuestra aplicación y los detalles de su diseño:

\begin{itemize}
	\item \textit{Pantalla principal:}
	
	\item \textit{Pantalla de destinos:}
	
	\item \textit{Pantalla de ruta:}
	
	\item \textit{Pantalla modo de uso:}
\end{itemize}

\subsection{Implementación}

En esta sección se exponen los detalles técnicos más relevantes. 

\begin{itemize}
	\item \textit{Conexión con el servidor}: la parte de la clase Cliente
	\item \textit{Seguimiento de la ruta}: la parte de onEddistonUpdate (o como se llame bien)
	\item \textit{Posicionamiento}: El tema de cómo se leen los \textit{beacons}
\end{itemize}
