\documentclass{article}
\usepackage[spanish]{babel}
\usepackage[utf8]{inputenc}
\usepackage{mathtools}
\usepackage{graphicx}
\usepackage[usenames]{color}
\usepackage{hyperref}


\title{\Huge Aplicaciones de guía para personas con discapacidad visual}

\begin{document}
	
	\begin{titlepage}
		\maketitle
		\thispagestyle{empty}
	\end{titlepage}
	
	%ABSTRACT
	\begin{abstract}
	    En los últimos años ha aumentado la concienciación acerca de la importancia de desarrollar tecnologías accesibles e inclusivas, de modo que, concretamente, cada vez son más las aplicaciones que tratan de reducir las limitaciones que antes las convertían en inalcanzables para personas con discapacidad visual.
	    
	    Entre ellas encontramos todo tipo de categorías: redes sociales, entretenimiento, lectura, identificación de colores y objetos, etc. 
	    
	    Nosotras haremos un pequeño estudio sobre qué apliaciones accesibles ya existen en el campo de la navegación, bien sea por interiores o exteriores, y cómo funcionan.
	
	\end{abstract}
	
	
	\section{Aplicaciones de guía para personas con discapacidad visual}
	
	%PRIMERA APP
	\subsection{Google Maps}
		El pasado 10 de Octubre de 2019, en el World Sight Day, Google dió a conocer la última actualización de la famosa aplicación "Google Maps". Esta incluiría una nueva característica desarrollada desde cero por y para personas con discapacidad visual que convertiría a la misma en una app accesible.
		\\
		\\
		El proyecto consiste en la implementación de una nueva funcionalidad, que facilita la posibilidad de recibir instrucciones de voz más detalladas y nuevos tipos de anuncios verbales muy útiles para las rutas de a pie.
		\\
		Algunas de las nuevas instrucciones incluídas son: informar de manera proactiva que estás en la ruta correcta, la distancia hasta el próximo giro, la dirección en la que estás caminando, avisos para cruzar con precaución si te aproximas a una gran intersección, notificaciones en caso de ser redirigido por causa de haber abandonado accidentalmente la ruta correcta, etc. De esta manera, la aplicación pretende brindar de independencia a las personas que padecen ceguera tratando de que se sientan cómodas y seguras a la hora de explorar lugares nuevos y desconocidos.
		\\
		\\
		La guía de voz detallada para la navegación está actualmente en desarrollo, estando ya disponible en inglés en los Estados Unidos y en japonés en Japón. Su soporte para otros idiomas y países sigue en camino.
		\\
		\\
		\textcolor{blue}{Link: https://blog.google/products/maps/better-maps-for-people-with-vision-impairments/}
		\\
		\\
        En cuanto a la navegación por interiores, Google Maps con su actualización 6.0 incorporó los primeros planos de ciertos edificios, entre los cuales destacan aeropuertos, centros comerciales, estadios y puntos de transporte público.
		\\
		Gracias a esta nueva versión, Google Maps ayuda a determinar dónde estás, en qué planta y hacia dónde ir. Para ello, basta con hacer zoom sobre un edificio cuyo plano esté disponible en la app, y éste aparecerá automáticamente y completamente detallado.
		\\
		\\
		A continuación vemos un ejemplo del famoso Madison Square Garden de Nueva York: 
		
		 \begin{figure}[h!]
			\centering
			\includegraphics[width=0.6\textwidth]{MadSq2}
			\caption{Plano del edificio. Arrastrando el muñeco "Google Street View" puedes posicionarte y visualizar el interior.}
			\label{fig:ejemplo}
		\end{figure}
		
		 \begin{figure}[h!]
			\centering
			\includegraphics[width=0.6\textwidth]{MadSq3}
			\caption{Vista del interior del Madison Square Garden. }
			\label{fig:ejemplo}
		\end{figure}
	    
		En el plano podrás localizar dónde están los baños, escaleras, ascensores, entradas y salidas, etc. los cuales aparecen representados mediante los iconos globalmente aceptados. También aparecen detallados los distintos establecimientos que se localizan en el edificio e incluye la posibilidad de hacer ciertas búsquedas, tanto generales (de cafeterías, librerías, tiendas, restaurantes...) como concretas (Starbucks, McDonald...). Otra funcionalidad que no falta en la versión de interiores es la posibilidad de señalar un destino y recibir indicaciones sobre cómo llegar a el. Para ello, aparece el habitual punto azul que te acompaña e indica tu posición, actualizando el plano con cada movimiento que lleves a cabo (incluidos cambios de una planta a otra).
		
		 \begin{figure}[h!]
			\centering
			\includegraphics[width=0.6\textwidth]{GMapsInd}
			\caption{Ejemplo de navegación y búsqueda en Google Maps Indoors. }
			\label{fig:ejemplo}
		\end{figure}
		Esta aplicación es un proyecto colaborativo y por ende, desde la web es posible actualizar y subir nuevos planos. Está disponible tanto para ordenador como plataformas Android e iOS.
		\\
		\\
		Pese al gran avance que supone en la navegación por interiores, cuenta con ciertas desventajas. El posicionamiento, al contrario que en exteriores, no es muy preciso y las búsquedas que puedes realizar son limitadas, no pudiendo, por ejemplo, preguntar por la ubicación de los baños. Pero, sobre todo tiene el inconveniente de que no es una tecnología accesible. Google Maps Indoors es una aplicación completamente visual que no cuenta con soporte auditivo por lo que descarta completamente a usuarios invidentes.
		\\
		\\
		
		\textcolor{blue}{Link: https://www.google.es/intl/es/maps/about/partners/indoormaps/
		\\
		Video: https://www.youtube.com/watch?v=cPsTWj.O3Qs El punto es un guión bajo, que petaba}	
		\\
	
	
		
	%SEGUNDA APP
	\subsection{BlindSquare}
	Es una de las aplicaciones de navegación más populares, su uso se extiende en más de 130 países y está habilitada en 25 idiomas, entre los cuales se incluye el español. Esta aplicación, desarrollada para iOs y diseñada para personas con discapacidad visual, proporciona una guía completa, de origen a destino, tanto en exteriores como en interiores. Además, describe el entorno y anuncia posibles puntos de interés para el usuario (como pueden ser los lugares considerados populares o aquellos visitados frecuentemente). Su principal característica es que permite interactuar mediante voz gracias al controlador de música de Apple. 
	\\
	\\
	BlindSquare determina tu posición mediante localización GPS y, a partir de ahí, puede darte información sobre las proximidades utilizando Foursquare y OpenStreetMap, de este modo, es capaz tanto de guiarte a un cierto destino como de notificarte qué establecimientos hay en tu radio: restaurantes a 200m, parques más cercanos, farmacias...
	\\
    Con el fin de agilizar el uso de la app y que por tanto, esta sea cómoda y rentable para los usuarios finales, incluye: accesos directos a funciones mediante gestos, como sacudir el móvil para que nos diga la ubicación actual y puntos cercanos; y, la posibilidad de establecer filtros para recibir únicamente información deseada. Filtrar por restaurantes para no tener notificaciones sobre estaciones de tren o librerías.
	\\
	\\
	En cuanto a la localización por interiores, BlindSquare emplea iBeacons y ¿¿VPS?? para solventar el problema del posicionamiento. Por lo demás, incluye las mismas posibilidades y funcionalidades que la navegación por exteriores, con la única limitación de que el edificio debe estar provisto de dichos sistemas de posicionamiento.
	\\
	\\
	Puntos fuertes de esta aplicación:
	\begin{itemize}
		\item Da información sobre los metros que quedan hasta llegar a un determinado objetivo. Resulta útil porque si van disminuyendo sabes que vas por el camino adecuado.
		\item Utiliza indicaciones reloj: a las 10, a las 3,...Muy utilizadas por las personas con discapacidad visual.
		\item Te avisa de las intersecciones. 
		\item Cuando te da una nueva indicación y la superas salta el sonido asociado a correto o check. Así, puedes seguir sin preocuparte. Si por el contrario te equivocas te salta un sonido en consecuencia.
		\item Se pueden añadir ubicaciones en una lista de lugares marcados.
		\item Puedes ir girando con el móvil y te va indicando lo que tienes enfrente. 
		\item También tiene opción de simulación, que permite prepararse un camino antes de ir.
		\item (Desde un ejemplo de la aplicación en uso) El usuario usaba Google Maps para llegar a cierta estación de metro pero se encontraba con indicaciones como " hacia el sureste", en ese momento abría BlindSquare para que se lo indicara con otro tipo de pauta.
		\item Te permite ser más autónomo y descubrir nuevos sitios.
		\item A la hora de desplazarte te indica las opciones por adelantado. Bus, metro, etc. para espacios exteriores o, escaleras, ascensor, escaleras mecánicas, etc. en el caso de interiores.
		\item Permite llevar las manos libres.
		\item Incluye un lector de códigos QR, es más cómodo porque puede dar más información que la línea braille.
	\end{itemize}

	Entre sus puntos negativos: cuesta 40 libras.
	\\
	\\
	\textcolor{blue}{Link iOS accesibility: 	https://developer.apple.com/accessibility/ios/
	\\
	Otras apps: https://www.henshaws.org.uk/wp-content/uploads/2017/06/24-Apps-eBook-1.pdf
	\\
	BlindSquare: https://www.blindsquare.com
	\\
	Video sobre BlindSquare: https://www.youtube.com/watch?v=ITw1Gs6tHLg
	\\
	Video de uso blindSquare en español y exterior: https://www.youtube.com/watch?v=i-5dosYeXw0
	\\
	Video de uso blindSquare en interiores: https://www.youtube.com/watch?v=9jH-Bdjmgb4
	\\
	Otro: el lector del QR en 5:40
	https://www.youtube.com/watch?v=AZUphzNVf48
	}
	
	\subsection{Nearby Explorer}
	Se trata de otra aplicación de navegación por GPS. Está disponible para Android e iOs y la podemos descargar desde el App Store de manera gratuita. 
	\\
	\\
	En cuanto a la guía por exteriores se comporta de manera similar a las dos aplicaciones de las que hemos hablando anteriormente. Destacando que permite explorar el entorno poniendo el móvil en distintas posiciones y, algunos de sus usuarios comentan, que es la aplicación que mejor da la información, puesto que puedes filtrarla y adaptarla a tus necesidades. Por ejemplo, permite escuchar información sobre lo que está cerca, los números de los bloques de las calles por los que se pasa, la distancia que hay al destino desde un punto de referencia como tu casa o el trabajo. Esto último es interesante, pues permite saber lo que te has alejado.
	También puedes pausar las indicaciones por audio de la aplicación en cualquier momento, por si necesitas atender a otras referencias sonoras (como las paradas en un autobús, por ejemplo), o explorar la ruta por adelantado, sin tener que estar físicamente en el sitio, o incrementar o decrementar el radio de exploración.
	\\
	\\
	Centrándonos ahora en la navegación por interiores: vemos que esta aplicación puede ser configurada de dos maneras: \textit{ad hoc} y \textit{mapeo completo}.
	
	Entre los problemas del uso de beacons \textit{ad hoc}:
	\begin{itemize}
		\item No se puede determinar la dirección de un beacon. 
		\item No se puede determinar qué está a tu alrededor a menos que pases cerca.
		\item Tienes que habilitar cierto soporte antes de detectar los beacons (no se detectan de manera automática).
	\end{itemize}

	Sin embargo, el \textit{mapeo completo} sí nos da una localización verdadera, es decir, la posición actual. Tiene un comportamiento similar al de las otras aplicaciones.
	
	Toda la info en:
	
	\textcolor{blue}{\href{https://developer.apple.com/accessibility/ios}{iOs}}
	
	\textcolor{blue}{\href{	https://play.google.com/store/apps/details?id=org.aph.nearbyonline&hl=es}{Nearby Explorer PlayStore}}
	
	\textcolor{blue}{\href{https://www.youtube.com/watch?v=Vlc0Pjv4Qro}{Video}}
	
	\textcolor{blue}{\href{https://www.youtube.com/watch?v=f2zE_yEL1Og}{ otro Video}}
	
	\textcolor{blue}{\href{https://www.youtube.com/watch?v=I_cJcOl5MPE}{tutorial}}
	
	
	
	\subsection{GetThere}
	Toda la info en: (ver txt no me deja pegar el link no sé por qué)
	 
	
	\subsection{Ariadne GPS}
	Toda la info en:
	\textcolor{blue}{\href{http://www.ariadnegps.eu/}{Ariadne GPS Info}}
 
	
\end{document}
© 2019 GitHub, Inc.
Terms
Privacy
Security
Status
Help
Contact GitHub
Pricing
API
Training
Blog
About