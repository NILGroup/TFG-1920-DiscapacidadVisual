\chapter{Conclusiones y Trabajo Futuro}
\label{cap:conclusiones}

\section{Conclusiones}

\section{Trabajo Futuro}

Tal y como está diseñada, la aplicación Blind Bit es muy fácilmente reutilizable en cualquier edificio público. Por lo tanto, animamos a proyectos futuros a que utilicen y extiendan la aplicación en espacios más ambiciosos como museos, aeropuertos, estaciones... en los que todo tipo de personas pueden encontrar muy útil esta ayuda. Además, proponemos una serie de pequeñas mejoras que pueden hacer que la app crezca:
\begin{itemize}
	\item Evaluación en usuarios finales invidentes para encontrar puntos mejorables de la app prestando especial interés en los aspectos relativos a la interfaz, a las instrucciones proporcionadas, en cómo de intuitiva resulta, en funcionalidades que puedan echar de menos, etc.
	\item Implementación de los ajustes de configuración, entre los que proponemos la posibilidad de cambiar el idioma, el tipo de voz que lee las instrucciones (mujer por defecto), el volumen, la activación o desactivación del modo\textit{ Instrucciones detalladas}, ect. 
	\item La inclusión de una base de datos que permita registrar y reconocer a distintos usuarios de manera que hagan \textit{Log In} y dentro de su sesión tengan una lista de destinos favoritos, sitios recientes, tengan guardada su configuración favorita en lugar de la por defecto, etc.
	\item Ampliación a una app más inclusiva de manera que no esté exclusivamente centrada en usuarios con discapacidad visual o videntes sino que este preparada y sea accesible para personas con otras discapacidades como por ejemplo la auditiva.
\end{itemize}


