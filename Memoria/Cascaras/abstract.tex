\chapter*{Abstract}

%\section*{\tituloPortadaEngVal}

\section*{}

More and more navigation applications are available. These are able not only to guide us between two points but also to give us information about the route, such as traffic congestion, bus schedules, etc. However, when our destination is inside a building the guidance options are scarce. The problem is even more complex when is the first time we go into the building, and there is some kind of visual disability that makes it difficult to read signs or other signage.

That is why in this work we present the development of an indoor guide application adapted to people with visual disabilities. This application tries to alleviate the feeling of disorientation that these users suffer when they have to visit for the first time a building that is unknown in their daily life. The School of Computer Science at the UCM has been used as a case study for this purpose. Thanks to the use of Bluetooth beacons, this application can determine our origin position within the building and guide us to our destination by means of intuitive and voice-adapted instructions, which also include additional information about the route (if there are points of interest nearby, such as toilets, a cafeteria, etc). Besides, the application is designed in such a way that it is easy to use for both visually impaired and sighted people, opening up the possibility of using it for the latter.

As shown in the evaluation of the application, it is completely general and independent of the building itself. Thus, the application can be deployed on any building you want, regardless of the number of floors or destinations you want to include.  In this way, all that is required is an installation of Bluetooth beacons and a mobile device with Android operating system to continue contributing to the adaptation of new spaces with the help of this project.

\section*{Keywords}

\noindent Indoor navigation, Bluetooth beacon, beacon, positioning, mapping, visual impairment, blindness, mobile application



