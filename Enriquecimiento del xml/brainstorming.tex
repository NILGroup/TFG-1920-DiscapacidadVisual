\documentclass{article}
\usepackage[spanish]{babel}
\usepackage[utf8]{inputenc}
\usepackage{mathtools}
\usepackage{graphicx}
\usepackage[usenames]{color}
\usepackage{hyperref}
\usepackage{parskip}

\title{\Huge Aplicaciones de guía para personas con discapacidad visual}
\author{Clara y Belén}

\setlength{\parskip}{12pt}
\setlength{\parindent}{12pt}

\begin{document}
	
	
	Este documento recoge algunas ideas para el enriquecimiento de los beacons y el posicionamiento con los mismos.
	
	\section{Cosas a incluir en los xml}
	
	\begin{itemize}
		\item El nombre del aula/estancia en la que nos encontremos. Por ejemplo, ``aula 5", ``pasillo 1".
		\item El beacon de ese cuadrante.
		\item Información sobre si en ese cuadrante hay un baño, una fuente,.. quizá para un ciego sea útil saber esa información. No sé si los baños también van a tener su propio beacon, deberían. 
		\item Añadir al xml las escaleras/ascensores (hay que tener en cuenta que no basta con indicar que hay escaleras, porque tienen descansillos y luego siguen, eso puede ir ad hoc).
		\item Incluir puertas e intersecciones porque van a llevar su propio beacon. 
		
	\end{itemize}

	\section{Casos de uso}
	Distintas situaciones que se pueden presentar:
	
	\begin{itemize}
		\item Estamos en un pasillo de aulas largo. 
		\item Acabamos de entrar por la puerta. 
		\item Acabamos de subir las escaleras.
		\item Estamos en una intersección (curva).
		\item Estamos en la puerta de la cafetería/biblioteca/conserjería/secretaría/aula....
		\item Estamos dentro de un aula.
		
	\end{itemize}
	
	
 
	
\end{document}
© 2019 GitHub, Inc.
Terms
Privacy
Security
Status
Help
Contact GitHub
Pricing
API
Training
Blog
About