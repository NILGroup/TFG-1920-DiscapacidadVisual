\documentclass{article}
\usepackage[spanish]{babel}
\usepackage[utf8]{inputenc}
\usepackage{mathtools}


\title{\Huge Entrevista en el CTI de la ONCE}
\date{11 de octubre 2019}

\begin{document}
	
	\begin{titlepage}
		\maketitle
		\thispagestyle{empty}
	\end{titlepage}
	
	%ABSTRACT
	\begin{abstract}
		La idea de este TFG surgió de la necesidad de resolver problemas reales para gente real, en este caso para personas con discapacidad visual.
		\\
		Como no podía ser de otra manera, el camino comienza por lo más importante: conocer las necesidades de los usuarios finales.
		\\
		Gracias a la oportunidad que nos  brindó la Universidad, con María Guijarro al frente, pudimos entrevistar a personas con discapacidad visual y especializadas en el ámbito de las tecnologías.
		\\
		En este documento recogemos las notas que tomamos el día de la entrevista en el CTI (Centro de Tiflotecnología e Innovación)  de la ONCE.
		
	\end{abstract}
	
	%PRIMERA SECCIÓN
	\section{Introducción}
	
	Empezamos la entrevista con una breve explicación sobre a qué está dedicado el centro. Entre algunas de sus tareas destacan:
	
	 \begin{itemize}
		\item Provisión de material necesario para la adaptación de puestos de trabajo (teclados, líneas de braille,...) y material para la vida diaria (bastones entre otros).
		\item Responder a consultas sobre el funcionamiento de dispositivos.
	\end{itemize}
	
	Dentro de este centro encontramos diversos departamentos:
	
	\begin{itemize}
		\item El departamento del chico de la presentación: donde se encargan del desarrollo de aplicaciones, como \textit{Erico???}, que es un editor de texto para matemáticas, o aplicaciones para la biblioteca de la ONCE, películas audiodescritas, etc.
		\item Departamento de Evaluaciones y Auditoría: donde se encargan de evaluar los productos que se van a sacar al mercado.
		\item Departamento de diseño y producción: donde se encargan de elementos de adaptabilidad como pueden ser plantillas con relieve de policarbonato para las vitrocerámicas, recordemos que estas no presentan gran dificultad para la mayoría de usuarios videntes pero sí para las personas con discapacidad visual porque no tienen ningún tipo de relieve donde ellos puedan marcar referencias.
		\item Departamento de Asesoría en tecnología: especializado en tecnología para personas con discapacidad visual.
	\end{itemize}
	
	
	%SEGUNDA SECCIÓN
	\section{Preguntas}
	
	Comienza la ronda de preguntas. Por turnos, los distintos grupos comenzamos a hacer preguntas a Mónica y (no me sé más nombres)
	\\
	\begin{itemize}
		\item  \textbf{¿Cómo utiliza una persona con discapacidad visual un móvil?}
		\\
		Para responder a esta pregunta, Mónica nos hace una demostración en directo.
		\\
		\\
		Ella usa un móvil Xiaomi con Android. Para desplazarse por las aplicaciones utiliza el \textit{flick}, un movimiento seco deslizando el dedo hacia uno de los lados de la pantalla (izquierda o derecha). Para la navegación por la web o dentro de alguna aplicación utiliza también estos movimientos hacia arriba y hacia abajo. Para seleccionar una opción lo que hace es dar un doble toque en la pantalla.
		\\
		También nos habla de la posibilidad de la navegación libre, cuando ya te has familiarizado con el dispositivo y sabes dónde tienes  determinadas aplicaciones es más rápido y cómodo acceder a ellas directamente sin tener que hacer un barrido por todas hasta llegar a una. Mónica, por ejemplo, agrupa las aplicaciones por carpetas, así se evita tener que hacer un barrido por todas ellas.
		\\
		\\
		Todos estos movimientos vienen acompañados de una voz que les indica en qué aplicación u opción se encuentran. En el caso de Mónica, esta aplicación es \textit{Talkback} de Google, es una aplicación de lectura de pantalla. 
		\\
		También hay otras posibilidades como la utilización de la linea braille conectada al dispositivo. 
		\\
		Para las personas con baja visión también existe la posibilidad de hacer más grandes los iconos y ajustar los colores.
		
		% Segunda pregunta:
		\item \textbf{¿Cómo se enfrenta una persona ciega al interior de un espacio cuando entra por primera vez?}
		\\
		La reacción a esta pregunta fue bastante autoexplicativa, Mónica resopló con un \textit{buufff...}. 
		\\
		Nos puso como ejemplo la llegada a un hospital: cuando entras necesitas saber, al menos, dónde está la recepción pero los carteles informativos están fuera de su alcance. Su sensación es \textit{¿Y ahora qué?}
		\\
		\\
		Ella nos comentó que en cuanto a la descripción/guía en espacios interiores ahora mismo no se dispone de ninguna aplicación, lo que nos animó aún más a continuar con este proyecto.
		\\
		\\
		Sería de gran ayuda para ellos el poder tener una idea del edificio incluso antes de llegar a él, para poder moverse con más seguridad. 
		\\
		Además, algo de gran ayuda sería no solo la guía que les indique dónde está traumatología, por ejemplo, sino dónde están los baños, la cafetería,...algo que te indique qué hay cerca.
		\\
		\\
		También se comentaron otras propuestas como la impresión de un mapa 3D que disponga de un código QR, por ejemplo, que tras leerlo de información sobre el espacio, esto incluye tanto información sobre el edificio en sí, número de plantas, qué hay en cada una...como información más precias como puede ser que uno de los baños esté averiado.
		\\ (Habría que investigar sobre la normativa de estos planos)
		 
	
	\end{itemize}


	%TERCERA SECCIÓN
	\section{Conclusiones}
	Algunas conclusiones que sacamos de la visita al CTI:
	
	\begin{itemize}
		\item La implementación de una aplicación como la nuestra hace falta.
		\item Tienen varias maneras de interactuar con el móvil: flick, sacudiéndolo, por vibración, arrastrando o pulsando con un dedo, dos,...
		\item Ir barriendo el espacio con la cámara del móvil no parece lo más adecuado porque no es un comportamiento habitual. Otra opción sería tomar una foto pero ¿cómo saben dónde enfocan?
		\item El uso de dispositivos adicionales como una micro cámara, en principio, no sería un problema, siempre y cuando no lo tengan que llevar siempre en la mano, pues esto les impide llevar el bastón o al perro cómodamente.
		\item Si se utilizan cascos, estos deben dejar el canal auditivo libre, para que puedan oír otros estímulos.
		\item Se trata de que las aplicaciones (en general) sean lo más inclusivas posibles, es decir, que el uso para personas videntes y no sea similar.
		\item El feedback que dé la aplicación no debe suponer un dolor de cabeza pero está bien que sea constante para que no piensen que la aplicación ha dejado de funcionar.
	\end{itemize}
	
	
	
\end{document}