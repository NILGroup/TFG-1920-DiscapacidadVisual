\chapter{Conclusions and Future Work}
\label{cap:conclusions}

This chapter sets out the conclusions reached after the project has been carried out and analysed. The conclusions of the project can be found in Section \ref{sec:concluFinales_eng}. The results of the project objectives are detailed in Section \ref{sec:descResult_eng} (see Section \ref{sec:objectives}) and finally, Section \ref{sec:trabajoFuturo_eng} opens the project to new collaborations with different proposals for future work.


\section{Conclusions}
\label{sec:concluFinales_eng}

The idea of this project arises from the need to empower people with visual impairment in the field of technologies. Even though today more and more people choose to develop accessible and inclusive applications for this group, there is still a considerable gap in the market for applications that are adapted for their use. For this reason, we have decided to study and research the field of accessible technologies and do our bit to help reduce the distances they face daily. To this end, our application constitutes a satisfactory solution to the problem of indoor navigation in general, and with a specific implementation for the School of Computer Science, perfectly adapted and inclusive.

The overall balance of the application is very positive as it meets the main objectives sought. Among its strengths we find:

\begin{itemize}
	\item It gives information about the meters left until you reach a certain point and as you move in the right direction they are updated decreasing.
	\item It uses simple directions: ``continue straight'', ``turn right (or left)'', etc.
	\item It warns of intersections using vibrations.
	\item It warns of the achievement of the final destination using vibrations.
	\item When a new indication is completed, the application emits the sound associated with correct or \textit{check}. In case the indication is not completed correctly, the app informs and recalculates the route.
	\item It has a \textit{detailed instructions} functionality that describes the user's environment allowing them to become familiar with it more quickly.
\end{itemize}



On the other hand, the great potential of this application lies in the fact that we have developed it independently from the original building so that, if certain guidelines for structuring the information are followed, it is possible to apply the guide to any building. To do this, it is necessary to have \textit{beacons} and distribute them inside the building, as they are the ones that determine the user's position. However, thanks to the positioning system used based on decision points, this installation is less rigid and therefore facilitates its integration into other buildings. 


\section{Descripción de los resultados obtenidos}
\label{sec:descResult_eng}



\section{Trabajo Futuro}
\label{sec:trabajoFuturo_eng}


