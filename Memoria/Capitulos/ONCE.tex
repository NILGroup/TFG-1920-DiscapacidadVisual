\chapter{ONCE}
\label{cap:once}

%LO DEJO AQUI, YA VEREMOS DONDE SE COLOCA
\section{Reunión en el Centro de Tiflotecnología e Innovación de la ONCE}

La idea de este trabajo de fin de grado surge de la necesidad de resolver problemas reales para gente real, concretamente para personas con discapacidad visual. De esta manera, empezamos nuestro camino por lo más importante: conocer las necesidades de los usuarios finales. Con este fin y gracias a la oportunidad que nos ha brindado la Universidad Complutense de Madrid con la profesora María Guijarro al frente, hemos podido reunirnos y entrevistar a personas especializadas en el campo de las tecnologías que sufren discapacidad visual.

En este documento recogemos las notas que tomamos durante la reunión el pasado 11 de octubre de 2019 en el CTI (Centro de Tiflotecnología e Innovación) de la ONCE, donde nos dieron una pequeña charla sobre la ceguera y las tecnologías accesibles que han surgido para reducir la brecha, y en la que finalmente conectamos con potenciales usuarios que nos hablaron sobre sus gustos y necesidades.


%PRIMERA SECCIÓN
\subsection{Introducción}
La visita al CTI comenzó con una breve explicación, de la mano de José María Ortiz, director del Departamento de Consultoría e Innovación, sobre las principales tareas que se llevan a cabo en el centro, entre las cuales destacan:

\begin{itemize}
	\item Ayudar a una persona con discapacidad visual en su \textbf{adaptación} al trabajo y a la vida cotidiana, proporcionandole para ello el material necesario (teclados, líneas de braille, bastones, etc.).
	\item Responder a \textbf{consultas} sobre el funcionamiento de dispositivos.
\end{itemize}

Luego, nos comentó los departamentos en los que se estructura el centro para que pudiésemos hacernos una idea más global de todo lo que abarca. Éstos son:

\begin{itemize}
	\item \textbf{El departamento de Consultoría e Innovación}, donde actualmente están desarrollando el programa EDICO (PONER REFERENCIA) en colaboración con la UCM, que tiene como objetivo hacer las matemáticas accesibles mediante un editor de texto. De manera paralela se encargan del desarrollo de aplicaciones de muy diversa índole, véase apps para la biblioteca de la ONCE, de películas audio-descritas, etc.
	\item \textbf{Departamento de Evaluaciones y Auditoría}, donde se encargan de evaluar los productos que se van a sacar al mercado.
	\item \textbf{Departamento de Diseño y Producción}, donde se encargan de, tal y como indica su nombre, diseñar y producir elementos de adaptabilidad, como pueden ser unas plantillas con relieve de policarbonato para las vitrocerámicas. Recordemos que estas, aunque no presentan dificultad alguna para los usuarios videntes, son tediosas para aquellos que cuentan con discapacidad visual ya que la pantalla táctil no tienen ningún tipo de relieve que pueda servirles como referencia y guiarles en su uso.
	\item \textbf{Departamento de Asesoría en Tecnología}, especializado en tecnologías accesibles.
\end{itemize}

Una vez concluida esta sección en la que nos contextualizaron, abrieron paso a la ronda de preguntas en la que pudimos acercarnos a ellos, conociendo sus problemas y necesidades en el día a día.

%SEGUNDA SECCIÓN
\subsection{Entrevista}

Durante esta parte, nos dirigimos especialmente a Mónica y José Luis Llorente, ambos ingenieros del CTI, para que, con su experiencia y conocimientos, nos explicaran lo máximo posible sobre tecnologías accesibles y nos dieran su punto de vista en las ideas que proponíamos. Por otro lado, Mónica no solo era experta en la materia sino que además es invidente, por lo que nos pudo contar su perspectiva y necesidades como usuaria.

Las preguntas avanzaron desde temas generales para conocer cómo una persona invidente se desenvuelve con la tecnología, sus gustos y qué sensaciones le despierta, hasta temas concretos dirigidos a conocer los problemas que plantea la navegación por espacios interiores:

\begin{itemize}
	\item  \textbf{¿Cómo utiliza una persona con discapacidad visual un dispositivo móvil?}
	
	Para responder a esta pregunta, Mónica nos hace una demostración en directo. Para ello emplea un móvil Xiaomi con sistema operativo Android.
	
	Mónica nos cuenta que para la navegación por su dispositivo utiliza un lector de pantalla, es decir, un software que facilita el uso del sistema operativo. Éste sirve como guía para las personas que, como ella, tienen discapacidad visual, ya que ``lee y explica'' mediante voz lo que se ve en la pantalla. Los lectores de pantalla vienen siempre incluidos en el dispositivo y se pueden encontrar en la sección de Accesibilidad, en Ajustes. En el caso de Android, este software se llama \textit{Talkback} y es configurable. Por ejemplo, dice Mónica, se podría usar mediante la línea de braille en vez de la reproducción por voz.
	
	Luego vemos como se desplaza por las aplicaciones utilizando \textit{flicks}, movimientos secos en los que desliza el dedo hacia uno de los lados de la pantalla (izquierda o derecha, según interese). Del mismo modo, para la navegación por la web o dentro de alguna aplicación utiliza estos movimientos hacia arriba y hacia abajo. Por último, nos muestra cómo accede a un elemento mediante doble click. 
	
	También nos habla de la posibilidad de la navegación libre, eso sí, solo cuando ya te has familiarizado con el dispositivo lo suficiente como para saber dónde tienes determinadas aplicaciones. 
	
	Lo más cansado, según Mónica, es tener que hacer un barrido por toda la pantalla hasta encontrar lo que quieres, en vez de poder ir directamente. Para agilizar un poco este proceso, Mónica, por ejemplo, agrupa las aplicaciones por carpetas, de modo que el barrido es más sencillo que si la pantalla estuviese repleta.
	
	Para las personas con baja visión también existe la posibilidad de hacer más grandes los iconos y ajustar los colores.
	
	% Segunda pregunta:
	\item \textbf{Hemos leído que normalmente las aplicaciones se desarrollan para dispositivos iOS, ¿por qué es mejor?} 
	
	\textit{``Si que es cierto que solía ser así ya que iOS le llevaba la delantera a Android en cuanto a accesibilidad, pero cada vez se usa más Android pues las diferencias están completamente recortadas, están muy igualados y los precios son mucho más asequibles. Yo misma antes tenía un iPhone y ahora me he pasado a Android y no hay nada que eche en falta.''}, responde Mónica.
	
	% Tercera pregunta:
	\item \textbf{¿Cómo afronta una persona ciega su desplazamiento y orientación por interiores cuando pisa por primera vez dicho espacio u edificio?}
	
	Ante esta pregunta Mónica resopla y nos contesta: \textit{``Buufff..., ¿te vale?''}
	
	Nos puso como ejemplo la llegada a un hospital: \textit{``cuando entras necesitas saber, al menos, dónde está la recepción para pedir ayuda pero los carteles informativos están fuera de mi alcance, entonces entro por la puerta y pienso ¿y ahora qué?. ¿Dónde está el mostrador de recepción? No es tan fácil como echar un primer vistazo, necesitas ayuda mediante voz, algo que te describa el espacio y te vaya diciendo que hay a derecha e izquierda y a cuantos metros.''}
	
	Nos contó que en cuanto a la descripción/guía por espacios interiores ahora mismo no hay disponible ninguna aplicación. Por ello, una vez superada la primera barrera de ubicar y localizar un cierto destino, la única opción que les queda es la de memorizar el camino. Mónica destacó que era increíble la cantidad de rutas que tiene en la cabeza.
	
	Por todo esto, se comentó que una aplicación del tipo que se quiere implementar en este trabajo de fin de grado sería de gran ayuda para ellos. No solo para que les guiase hasta un punto concreto, sino para que también describiese el edificio, facilitándoles una primera idea del mismo que les ayudase a moverse con mas seguridad y les indicase qué posibilidades les ofrece. En este punto se mencionaron otras propuestas e ideas sobre cómo proporcionar esta información estática del edificio. Algunas de ellas fueron: 
	
	\begin{itemize}
		\item[$-$] Tener subido el plano de las distintas plantas del edificio e implementar un sistema de manera que cuando se deslice el dedo sobre las distintas salas que aparecen la app indique cuales son (aula X, cafetería, secretaría, pasillo, etc.).
		\item[$-$] La impresión de un mapa 3D que disponga de un código QR o algo similar que sea capturado por Bluetooth (mejor que por foto) y que tras leerlo cargue el plano del edificio y pueda proporcionar tanto información sobre el espacio en sí mismo (número de plantas, qué hay en cada una...) como información cambiante como la existencia de averías, horarios, disponibilidad de salas, etc.
	\end{itemize}
	
	Como es natural, de la mano de estas ideas surgieron problemas y opiniones a favor y en contra: ¿Dónde estaría dicho mapa?, ¿Cómo encontrarlo?, ¿Todos los edificios estarán de acuerdo en facilitar los planos o puede que por motivos de seguridad no sea una idea factible?, ¿Es posible llegar a un standard para que se pueda usar el mismo sistema en cualquier edificio?
	
	% Cuarta pregunta:
	\item \textbf{¿Hay algún tipo de señales que os sirvan como referencia a la hora de desplazaros por un edificio?}
	
	\textit{``Hay señales de encaminamiento, que te indican dónde están las escaleras, ascensores, zonas de cruce, etc.''}, contesta.
	
	% Quinta pregunta:
	\item \textbf{¿Cuántos edificios cuentan con estas señales?}
	
	\textit{``La verdad que cada vez son más frecuentes y hoy en día se encuentran en casi todos los edificios, especialmente en los nuevos.''}, responde.
	
	% Sexta pregunta:
	\item \textbf{¿Cómo de factible es ir con el dispositivo móvil en la mano, para realizar una foto o cualquier cosa similar?}
	
	\textit{``Puedo hacer una foto en un momento puntual, en eso no hay problema alguno pero no es cómodo ir con el móvil en la mano constantemente porque además de que es aparatoso porque ya llevo en la mano el bastón, perro guía, etc. No es práctico, no sería la primera vez que roban un móvil a una persona invidente, es una realidad.''}, contesta Mónica. \textit{``Particularmente, con respecto a la foto el problema principal sería saber a dónde enforcar''}, añade.
	
\end{itemize}

%Creo que es mejor subsection para las conclusiones del TIC
%TERCERA SECCIÓN
\subsection{Conclusiones}
Tras el debate, algunas de las conclusiones que sacamos de la visita al CTI son:

\begin{itemize}
	\item La implementación de una aplicación como la nuestra es muy útil y necesaria.
	\item Las modalidades más empleadas para interactuar con el móvil cuando tienes algún tipo de deficiencia visual son: flicks, sacudidas, mediante vibración, arrastrando o pulsando la pantalla con un dedo, dos,...
	\item No resulta cómodo ir barriendo el espacio con la cámara del móvil.
	\item El uso de dispositivos adicionales como una micro cámara, en principio, no sería un problema, siempre y cuando no la tengan que llevar de manera continuada en la mano.
	\item En caso de auriculares, se recomienda utilizar auriculares óseos de modo que dejen el canal auditivo libre para captar otros estímulos.
	\item El objetivo es que el grueso de las aplicaciones sean lo más inclusivas posibles, es decir, que su uso sea apto tanto para personas videntes como invidentes.
	\item El feedback de la aplicación no debe saturar pero sí se aconseja que sea constante para que no se malinterprete que la aplicación ha dejado de funcionar.
\end{itemize}
