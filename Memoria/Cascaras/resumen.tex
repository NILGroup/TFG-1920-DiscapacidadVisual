\chapter*{Resumen}

\section*{\tituloPortadaVal}

%Un resumen en castellano de media página, incluyendo el título en castellano. A continuación, se escribirá una lista de no más de 10 palabras clave.


Cada vez son más las aplicaciones de navegación disponibles, capaces no solo de guiarnos entre dos puntos, sino  también de darnos información sobre el trayecto, como la congestión del tráfico o los horarios de un autobús. Sin embargo, cuando nuestro destino está en el interior de un edificio las opciones de guía son escasas. El problema es aún más complejo cuando el edificio no es conocido y se dispone de algún tipo de discapacidad visual, que dificulte la lectura de carteles u otras señalizaciones. Es por ello que a lo largo de este documento se expone el desarrollo de una aplicación de guía en interiores adaptada a personas con discapacidad visual. A través de la cual se intenta paliar la sensación de desorientación de estos usuarios en estas situaciones tan frecuentes en el día a día. Gracias a la tecnología Bluetooth, esta aplicación es capaz de determinar nuestra posición origen dentro del edificio y guiarnos hasta el destino mediante instrucciones intuitivas y adaptadas por voz, que incluyen también información adicional sobre el recorrido que se va haciendo (si hay puntos de interés relevantes cerca, como aseos, una cafetería, etc). Además, la aplicación está diseñada de tal manera que el manejo de la misma sea sencillo tanto para personas invidentes como videntes, abriendo a estos últimos la posibilidad de utilizarla.


Gracias a la generalidad y la tecnología con la que la aplicación ha sido implementada, esta puede desplegarse sobre cualquier edificio que se desee, independientemente del número de plantas o destinos que se quiera incluir.  De esta manera, todo lo que se requiere es balizas Bluetooth y un dispositivo móvil con sistema operativo Android para continuar contribuyendo a la adaptación de nuevos espacios con la ayuda de este proyecto.

\section*{Palabras clave}
   
%\noindent Máximo 10 palabras clave separadas por comas
\noindent Navegación, interiores, Bluetooth, baliza, beacon, posicionamiento, mapeo, discapacidad, visual
   


